\section{Lineare Regression}
\label{sec:LineareRegression}
Hier im Wesentlichen Stroetmann, das ist denke ich das beste. 
Was ist das und was macht es, warum ist das erstes Kapitel
\subsection{Konzept und Ziele linearer Regression}
Wofür brauche ich das, was kann ich damit machen, was kann ich damit nicht machen?
\subsubsection{Beispiel}
z.B. Beispiel mit Gerade durch Punkte ziehen, Beispiel sollte für einfache und allgemeine Lineare Regression brauchbar sein 

Tabelle aus Werten, damit man später Funktion plotten kann udn mehr Ressourcen hat 
\subsubsection{Arten von Bias}
Hier kurzer Text, was ein "Bias" ist, und danach Erklärungen der Abweichungen \cite{BiasTypes}
\paragraph{Bias aus Varianz} Grundlegende Abweichungen der Sache aus harten Gründen - etwa Schwankungen im Wetter die einfach auftreten können

\paragraph{Selection Bias} Abweichung wenn man seltsame/dumme Stichproben nimmt, z.B. Ernährungsumfrage auf Veganermesse

\paragraph{Confirmation Bias} Wenn man (unterbewusst) Werte nimmt, die eine gewisses Schema erfüllen 
\subsection{Einfache Lineare Regression}
Hier ist Lineare Regression auf einzelne Werte also 

$R^1 -> R^1$

\subsection{Allgemeine Lineare Regression}
Hier ist die komplizierte Regression gemeint, wie wir sie brauchen also

$R^n -> R^m$

mit vielen Vektoren, Matrizen und tollen Dingen

\subsection{Bewertung der Linearen Regression}
Wie berechne ich die statistische Signifikanz meines Linearen Modells?