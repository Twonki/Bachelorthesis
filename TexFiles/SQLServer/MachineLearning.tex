\section{Machine Learning im SQL-Server 2017}
\label{sec:MLSQL} \label{sec:MachineLearning}
Innerhalb dieses Abschnittes befinden sich Code-Beispiele zur Umsetzung der in Kapitel \ref{cha:Theorie} vorgestellten Algorithmen. 

Es werden direkt SQL-Befehle 
\paragraph{Möglichkeiten in  R}
Die Sprache R besitzt verschiedene Optionen Machine-Learning Modelle zu erzeugen. Neben der Implementation \textit{von Grund auf} gibt es ebenfalls eine Vielzahl von Paketen und Bibliotheken. 

Für die lineare und logistische Regression wird die Bibliothek \textit{RevoscaleR} von Microsoft benutzt (Die Dokumentation findet sich unter \cite{RevoscaleR}). Sie wird bereits mit dem SQL-Server geliefert. 

Für die Neuronalen Netze wird das Paket \textit{RSNNS} benutzt. Die vollständige Dokumentation findet sich unter \cite{RSNNSDoku}. Neben Umsetzungen von \textit{Computational Networks} befinden sich in dem Paket ebenfalls Funktionen zum Test der Modelle.
\paragraph{Möglichkeiten in Python}
Der SQL-Server 2017 unterstützt neben einem R-Server ebenfalls eine Instanz des Microsoft ML-Servers. Dieses Open Source Projekt zu finden auf Github \cite{GithubMLServer} stellt eine Alternative zu den in R vorgestellten Methoden dar. 

Der ML-Server ist in Python implementiert und die Benutzung verhält sich ähnlich zu den Ansätzen von Tensorflow in Python. 

\paragraph{Verwendung von R im SQL-Server}
Hier zeige ich kurz, wie man ein einfaches R Skript in ner Stored Procedure aufruft
\subsection{Lineare Regression}
Bedingungen an Lineare Regression in R

Auflösen des Algorithmus in R (Programmcode)

Vllt Hintergrundwissen/Parameter wenn möglich mit Erklärung. 
\subsection{Klassifikation}
Wie Lin. Regression
\subsection{Neuronale Netze}
\paragraph{Von Grund auf}
Hier quasi den Code und Beispiel von \cite{SelbyNN} übernehmen und etwas aufbereiten.

NICHT BEHANDELN! UMFANG!

\paragraph{Mit Package NNet}
Ich denke, ich sollte das Package benutzen. Das haben totale Profis geschrieben. 