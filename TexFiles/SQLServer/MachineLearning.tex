\section{Machine Learning im SQL-Server 2017}
\label{sec:MLSQL} \label{sec:MachineLearning}
Innerhalb dieses Abschnittes befinden sich Code-Beispiele zur Umsetzung der in Kapitel \ref{cha:Theorie} vorgestellten Algorithmen. 

Es werden im Folgenden kurz die Einbindung der R-Skripte in TSQL behandelt, anschließend werden nur die R-Skripte für die einzelnen Punkte erläutert.
\paragraph{Möglichkeiten in  R}
Die Sprache R besitzt verschiedene Optionen Machine-Learning Modelle zu erzeugen. Neben der Implementation \textit{von Grund auf} gibt es eine Vielzahl von Paketen und Bibliotheken. 

Für die lineare und logistische Regression wird die Bibliothek \textit{RevoscaleR} von Microsoft benutzt (Die Dokumentation findet sich unter \cite{RevoscaleR}). Sie wird bereits mit dem SQL-Server geliefert. 

Für die Neuronalen Netze wird das Paket \textit{RSNNS} benutzt. Die vollständige Dokumentation findet sich unter \cite{RSNNSDoku}. Neben Umsetzungen von \textit{Computational Networks} befinden sich in dem Paket ebenfalls Funktionen zum Test der Modelle.
\paragraph{Möglichkeiten in Python}
Der SQL-Server 2017 unterstützt neben einem R-Server ebenfalls eine Instanz des Microsoft ML-Servers. Dieses Open Source Projekt zu finden auf Github \cite{GithubMLServer} stellt eine Alternative zu den in R vorgestellten Methoden dar. 

Der ML-Server ist in Python implementiert und die Benutzung verhält sich ähnlich zu den Ansätzen von Tensorflow in Python. 

\paragraph{Verwendung von R im SQL-Server}
Um R im SQL-Server zu benutzen wird die Stored Procedure \textit{sp\_execute\_external\_script} benötigt. Im Folgenden ein einfaches Beispiel: ~\newline

\begin{lstlisting}[language=SQL]
	EXECUTE sp_execute_external_script
	@language = N'R',
	@script = N' 
		mytextvariable <- c("hello", " ", input_data);
		OutputDataSet <- as.data.frame(mytextvariable);',
	@input_data = N' SELECT name FROM readers'
	WITH RESULT SETS (([Greetings] char(20) NOT NULL));
\end{lstlisting}

Hierbei wird in Zeile 2 zunächst die Sprache als Parameter übergeben, in Zeile 4 wird innerhalb des R Skriptes ein Begrüßungs-String erstellt, welcher in Zeile 5 als Ausgabe wiedergeben wird.

In Zeile 6 wird die Inputvariable definiert, an dieser Stelle sind SQL Befehle und gültige T-SQL Variablen möglich. Es können beliebig viele Inputvariablen definiert werden. 

In Zeile 7 wird die Ausgabe in Tabellenform normiert. Diese Zeile ist nicht notwendig.  

~\newline Dieses Schema bleibt allen Skript-Aufrufen gleich. Im Folgenden werden nur die R-Skripte vorgestellt.
\subsection{Lineare Regression}
Für die diese Form der Regression gelten innerhalb des Paketes RevoScaleR folgende Bedingungen: ~\newline
\begin{enumerate}
	\item Alle Eingabewerte des Modells müssen (reelle)\footnote{Es gibt andere Pakete, die komplexe Zahlen unterstützen} Zahlen sein
	\item Texteingabewerte müssen zuvor über ein Dictionary realisiert werden
	\item Der Ausgabewerte ist eine reelle Zahl 
\end{enumerate}

Um ein Modell für die lineare Regression zu erstellen, sind in R nur wenige Zeilen notwendig: ~\newline
\begin{lstlisting}[language=R]
	formel <- C ~ A+B;
	model <- rxLinMod(formula=formel, data=TrainingsData);
	serializedModel <- data.frame(payload = as.raw(serialize(model, connection=null)));
\end{lstlisting}

In der ersten Zeile wird zunächst eine allgemeine Formel definiert. Diese Formel ist zu interpretieren als $f: (A~x~B)\rightarrow C $ , das '+' ist hierbei nicht als Addition zu verstehen.

In Zeile 2 wird das Modell mithilfe der Bibliothek RevoscaleR und dem Methodenaufruf rxLinMod erstellt \textbf{und} Trainiert. Als Parameter werden die Formel und die Trainingsdaten benötigt. 

In der dritten Zeile findet eine Serialisierung des Modells statt - dies ist nicht notwendig für eine direkte Verwendung, ermöglicht allerdings das speichern des Modells innerhalb des SQL-Servers als Blob.  

Um das Modell zu benutzen reichen ebenfalls wenige Zeilen R-Skript: \newline

\begin{lstlisting}[language=R]
	model <- unserialize(as.raw(serializedModel)); 
	C <- rxPredict(model,data.frame(TestData));
\end{lstlisting}

Hierbei wird zunächst in Zeile 1 das serialisierte Modell wieder nutzbar gemacht. 

In Zeile 2 wird die Methode \textit{rxPredict} der RevoScaleR-Bibliothek aufgerufen, welche aus den zu testenden Daten und dem Model eine Prognose erstellt. 
\subsection{Klassifikation}
Für die Klassifikation mit RevoscaleR gelten folgende Bedingungen: ~\newline

\begin{enumerate}
	\item Alle Eingabewerte des Modells sind reelle Zahlen
	\item Texteingabewerte müssen zuvor über ein Dictionary realisiert werden
	\item Der Ausgabewerte ist eine Zahl zwischen 0 und 1, welche die Wahrscheinlichkeit 
	\item Es kann gleichzeitig nur eine Klasse überprüft werden
\end{enumerate}

Der R-Code verhält sich parallel zum Code der linearen Regression:

\begin{lstlisting}[language=R]
	formel <- rain ~ temperature+humidity;
	logitmodel <- rxLogit(formula = form, data = TrainingsData);
	rainPropability <- rxPredict(model, data.frame(TestData));
\end{lstlisting}

Als Beispiel wurde hierbei die Voraussage gewählt, ob es regnet anhand von Temperatur und Luftfeuchtigkeit.
\subsection{Neuronale Netze}
Vorsatz?
\paragraph{Von Grund auf}
Es ist Möglich, die im Abschnitt \ref{sec:NN} vorgestellten Konzepte direkt in R umzusetzen. Ein gutes Tutorial liefert hierbei \cite{SelbyNN}, welcher eine Schritt-Für-Schritt Anleitung und Erklärung bietet ein eigenes Neuronales Netz zu entwerfen. Das Tutorial von Selby setzt einen ähnlichen Blogeintrag von \cite{DennyNN} in R um. 

Innerhalb dieser Arbeit wird allerdings das Paket \textit{RSNNS} verwendet. Hauptgründe hierfür sind die größeren Funktionsumfänge, die Auswahl mehrerer Computational Networks sowie die Dokumentation.
\paragraph{Mit Package RSNNS}
Ich denke, ich sollte das Package benutzen. Das haben totale Profis geschrieben. 

\improvement{Check this if it works ! rxNNEt https://cloudblogs.microsoft.com/sqlserver/2017/01/05/how-six-lines-of-code-sql-server-can-bring-deep-learning-to-any-app/}