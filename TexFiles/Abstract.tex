\chapter*{Abstract} %*-Variante sorgt dafür, das Abstract nicht im Inhaltsverzeichnis auftaucht
This Thesis gives an overview and examples about machine learning in SQL Server 2017 using Neural Networks, with the goal to predict use-cases regarding NYC-Taxidata, such as predicting the tip-amount of a ride or estimate total rides for a certain location and hour.  

~\newline Goals of this thesis are to explain main-components necessary for Neural Networks as well as its use and representation in the SQL-Server with the programming language R. 

~\newline Contents of this work cover general regression and classification as basis for neural networks, explanation and examples for the implementation of neural networks in the SQL Server with R, a case study of several usecases using NYC Taxi-Data including performancetuning and summarized best practices working with neural networks in R. 
~\newline
~\newline
\begin{flushleft}
	\begin{tabular}{ll}
		\textbf{title:} &\quad Machine-Learning and Prognosis \\
		\textbf{author:}  &\quad Leonhard Applis \\
		\textbf{matriculation number:} &\quad 2086307 \\
		\textbf{class:} &\quad TINF15/AI-BI \\
		\textbf{supervisor DHBW:} &\quad \betreuerdhbw \\
		\textbf{supervisor Atos:} & \quad \betreuerfirma \\
		[6ex]%formerly 5ex
	\end{tabular} 
\end{flushleft}


\chapter*{Kurzfassung} 
Diese Bachelorarbeit gibt einen Überblick und Beispiele für Machine-Learning im SQL Server 2017 unter Verwendung Neuronaler Netze, mit dem Ziel verschiedene Prognosemodelle für ein New-Yorker Taxiunternehmen zu erstellen, wie etwa die Vorhersage des Trinkgeldes einer einzelnen Fahrt oder die Schätzung des Fahrtenaufkommens einer bestimmten Gegend. 

~\newline Ziel dieser Arbeit ist eine Vorstellung der Hauptkomponenten eines Neuronalen Netzes sowie seiner Repräsentation, Erstellung und Benutzung innerhalb des SQL Servers mit der Programmiersprache R.  

~\newline Inhalte dieser Arbeit sind Regressions- und Klassifikationsalgorithmen als Basis zur Erklärung neuronaler Netze, Erklärungen und Implementation von neuronalen Netzen im SQL-Server mit R, eine Fallstudie anhand mehrerer Use-Cases für New Yorker Taxidaten inklusive Performanzverbesserungen und Best-Practices bei der Arbeit mit neuronalen Netzen im SQL-Server.
~\newline
~\newline
\begin{flushleft}
	\begin{tabular}{ll}
		Titel:& \quad \titel \\ 
		Author:& \quad Leonhard Applis \\
		Matrikelnummer: & \quad \matrikelnr  \\
		Kurs: & \quad \kurs \\ 
		Betreuer der Dualen Hochschule: & \quad \betreuerdhbw \\ 
		Betreuer der Firma: & \quad \betreuerfirma \\
		[6ex]%formerly 5ex	
	\end{tabular} 
\end{flushleft}