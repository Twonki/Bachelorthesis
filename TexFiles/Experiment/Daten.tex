\section{Eigenschaften der Daten}
\label{sec:Daten}
Innerhalb dieses Abschnittes werden zunächst die Daten vorgestellt, die dem Fallbeispiel zugrunde liegen. 

Die vorgestellten Daten haben bereits einen ETL-Prozess durchlaufen. Dieser besteht im Wesentlichen darin, die CSV-Dateien dahingehend aufzubereiten, das amerikanische Nummerierungen (z.B. Angabe von Dezimalzahlen mit '.' anstelle von ',') auf europäische Normen gebracht werden. Prinzipiell entfällt dieser Schritt für eine rein amerikanische Umgebung. 
\subsection{Taxifahrten}
\label{subsec:Taxidaten}
Zunächst werden die Daten der Taxifahrten erläutert. 

~\newline Diese Daten stammen von der Stadt New York \cite{SourceTaxi} und wurde von der \textit{Taxi and Limousine Commission} (Kurz: TLC) bereitgestellt. 

Die TLC stellt einen Dachverband mehrerer Taxiunternehmen dar und veröffentlicht die Daten - die Erhebung erfolgt in einzelnen, anonymisierten Kleinunternehmen. 

~\newline Zusätzlich teilen sich die Fahrten in zwei Kategorien auf: \textit{Green} und \textit{Yellow}. Bei grünen Fahrten handelt es sich um Fahrzeuge mit einer anderen Lizenzierung (vgl. \cite{GreenTaxis} Absatz 5ff) und besonderen Auflagen. Im Allgemeinen verhalten sich die Fahrten allerdings gleich, insofern werden lediglich Unterschiede aufgelistet, falls diese bestehen.
\paragraph{Attribute und Datentypen} ~\newline
Die folgende Übersicht entspricht der von der NYC bereitgestellten Übersicht (Siehe \cite{DataDicYellow}), die Beschreibung wurde übersetzt und eine Spalte für den Datentyp \footnote{Wie sie innerhalb des SQL-Servers bezeichnet werden} ergänzt. 
~\newline
\begin{center}
	\begin{tabular}{|p{0.3\textwidth}|p{0.5\textwidth}|p{0.1\textwidth}|}
		\hline
		Name & Beschreibung & Datentyp  \\ \hline
		\textbf{VendorID} & Ein Code für das Taxiunternehmen, welches die Daten bereitstellt & smallint \\ \hline
		pickup\_datetime & Uhrzeit und Datum, wann die Fahrt begann & datetime \\ \hline
		dropOff\_datetime & Uhrzeit und Datum, wann die Fahrt endete & datetime \\ \hline
		Passenger\_count & Anzahl der Fahrgäste & smallint \\ \hline	
		store\_and\_fwd\_flag & Angabe, ob die Fahrt direkt hochgeladen wurde, oder ob die Fahrt zwischengespeichert wurde vor einem Upload & bit \\ \hline
		RatecodeID & Ein Code für die Rate, welche für die Taxifahrt bezahlt wurde & smallint \\ \hline
		PULocationID & Ein Code für die Zone, in welcher die Fahrt begann & smallint \\ \hline
		DOLocationID & Ein Code für die Zone, in welcher die Fahrt endete & smallint \\ \hline
		trip\_distance & Distanzangabe des Taximeters & real \\ \hline
		fare\_amount & Der Fahrpreis berechnet aus Zeit und Distanz & \\ \hline
		extra & Verschiedene Zuschläge auf den Fahrpreis & real \\ \hline 
		MTA\_tax & Aufschlag, automatisch erhoben bei entsprechender Rate& real \\ \hline
		improvement\_surcharge & Aufschlag, automatisch erhoben in bestimmten Zonen & real \\ \hline
		payment\_type & Angabe des Zahlungsmittels als Code & smallint \\ \hline 
		tip\_amount & Höhe des Trinkgeldes & real \\ \hline
		tolls\_amount & Summierter Betrag von Zuschlägen dieser Fahrt& real \\ \hline
		total\_amount & Gesamtbetrag der Fahrt \textbf{ohne Trinkgeld} & real \\ \hline 
	\end{tabular}
\end{center}
~\newline
Die Daten der Grünen Taxis sind erweitert um einen Code für den \textit{Trip\_Type} (Ob eine Fahrt von einem Taxistand begann oder ob die Gäste an der Straße abgeholt wurden).

~\newline
Alle Distanz-Angaben entsprechen amerikanischen Meilen (1 mile$ \rightarrow $1,6 km), alle Währungsangaben Dollar. 

Für die Angaben der Codes sind ebenfalls Dictionaries bereitgestellt, diese spielen allerdings für den Machine-Learning-Aspekt dieser Arbeit keine Rolle.
\paragraph{Umfang} ~\newline
Aus Ressourcengründen wurde ausschließlich das Jahr 2017 betrachtet. 

~\newline Es gibt \textbf{113 Millionen} Einträge für gelbe Fahrten, welche insgesamt knapp \textbf{8,1 GB Speicher} benötigen. Zusätzlich wurden Indizes angelegt mit weiteren 9,4 GB Speicher (Um schnelle Anfragen auf Uhrzeiten und Orte zu ermöglichen).

~\newline Es gibt \textbf{11,7 Millionen} Einträge für grüne Fahrten, mit insgesamt \textbf{900 MB Speicherplatz}. Es wurden zusätzlich Indizes mit 1,1 GB Speicher erstellt. 

~\newline Zusammen gibt es aus dem Jahr 2017 also fast \textbf{125 Millionen Einträge}, welche insgesamt 19,5 GB Speicher belegen.
\paragraph{Anomalien} ~\newline
Innerhalb der Daten traten einige Ungewöhnlichkeiten auf - zum Beispiel gibt es Fahrten, die von Ort A nach Ort A verliefen und keine Strecke zurückgelegt haben. Bei einer genaueren Untersuchung ergab sich allerdings, dass diese Fahrten meist wenige Minuten dauerten und ebenfalls keinen Passagier hatten. 

~\newline Es ist anzunehmen, das die Taxis an dieser Stelle auf ihre Passagiere gewartet haben. Zunächst wurde versucht, die Anomalien mit in die Machine-Learning-Algorithmen aufzunehmen. Erste Ergebnisse zeigten allerdings desaströse Resultate, welche v.a. daher rührten, dass extreme Reichweiten der Eingabedaten auftraten. 

Es wurden außerdem weitere Anomalien gefunden, welche kurz genannt werden:
\begin{itemize}
	\item Fahrten mit Negativkosten
	\item Fahrten außerhalb von 2017
	\item Fahrten mit extrem hohen Trinkgeld ($\sim 100$\$) oder extrem hohen kosten ($\sim300$\$)
	\item Fahrten, die wenige Sekunden gedauert haben
	\item Fahrten, welche mehrere Stunden gedauert haben und dabei nur kurze Strecken zurücklegen
\end{itemize}

Die Daten wurden nach Maßgabe des Autors auf \textit{gesunde} Werte gekürzt (keine Fahrten über 150\$ , nicht mehr als 40\% Trinkgeld, nur positive Kosten, nur Fahrten in 2017).
\subsection{Wetteraufzeichnungen}
\label{subsec:Wetterdaten}
In diesem Unterabschnitt werden die Wetterdaten sowie ihr Umfang vorgestellt. 

~\newline Die Wetterdaten stammen von der \textit{National Oceanic and atmospheric Administration} \cite{SourceWeather} (Kurz: NOAA), welche verschiedene Klimadaten sammelt. Für dieses Fallbeispiel wurden die Wetterdaten der Wetterstation des JFK-Airports für das Jahr 2017 abgefragt. 

\paragraph{Attribute und Datentypen} ~\newline
Im Gegensatz zu den Taxidaten werden in diesem Paragraphen lediglich die verwendeten Attribute vorgestellt. Es werden ebenfalls der Bezeichner, eine kurze Beschreibung und der Datentyp innerhalb des SQL Server vorgestellt. 

~\newline
\begin{center}
	\begin{tabular}{|p{0.3\textwidth}|p{0.5\textwidth}|p{0.1\textwidth}|}
		\hline
		Name & Beschreibung & Datentyp  \\ \hline
		\textbf{Date} & Der Tag, an welchem der Datensatz erhoben wurde & date \\ \hline
		\textbf{Hour} & Die Stunde, an welcher der Datensatz erhoben wurde & smallint \\ \hline
		\textbf{DryBulbTemp} & Die Trockenkugeltemperatur & real \\ \hline
		\textbf{WetBulbTemp} & Die Feuchtkugeltemperatur (Messung unter Berücksichtigung von Verdunstungskälte) & real \\ \hline	
		\textbf{DewPointTemp} & Die Höhe des Taupunktes & real\\ \hline
		\textbf{RelativeHumidity} & Die gemessene Luftfeuchtigkeit & real \\ \hline 
		\textbf{Visbility} & Sichtweite in Meilen & real\\ \hline
		\textbf{WindSpeed} & Windgeschwindigkeit & real\\ \hline
		\textbf{WindDirection} & Windrichtung in Grad & int \\ \hline
		\textbf{Sunrise} & Uhrzeit des Sonnenaufgangs & datetime \\ \hline
		\textbf{Sunset} & Uhrzeit des Sonnenuntergangs & datetime \\ \hline
	\end{tabular}
\end{center}
~\newline
Die Windgeschwindigkeiten sind hierbei in Meilen/Stunde angegeben, die Temperaturen in Grad Celcius auf eine Nachkommastelle gerundet. Die Windrichtung ist als Grad angegeben, wobei $360^\circ$ Norden und $180^\circ $ Süden entsprechen.

~\newline Es gab keine nennenswerten Anomalien. 
\paragraph{Umfang} ~\newline
Es gibt \textbf{13351 Datensätze} die 1,4 MB Speicher benötigen. Zusätzlich gibt es 0,6 MB Indizes.

\subsection{Machine-Learning-Sicht und Rich-Sicht}
Die in den vorhergehenden Unterabschnitten vorgestellten Daten sind für die Verwendung als Sicht zusammengefasst, so das die Taxifahrten um die jeweiligen Wetterdaten angereichert werden. 

Insgesamt wurden vier Sichten erstellt, je zwei für die grünen und gelben Taxis:

\paragraph{Rich-View}
Enthält alle Daten in lesbarer Form, Locations wurden nach \textit{Borough} und \textit{Zone} aufgeteilt. In gleichem Maße sind die HändlerID's, Zahlungsmittel und Raten als Text aufgelöst.

Diese Sicht wurde erstellt, um Anomalien zu erkennen und den Import der Daten zu überprüfen. Für Machine Learning ist Sie im Allgemeinen unbrauchbar. 

\paragraph{Machine-Learning-View}
Enthält die für das Machine Learning benötigten Trainingsdaten. Jeder Satz der gefilterten Taxi-Daten erhält eine Erweiterung um die aktuell herrschenden Wetterdaten. 

Die Orte, Raten und Zeiten liegen als numerischer Faktor vor. 

\paragraph{Ausschnitt-Tabellen}
Neben den Views werden des weiteren kleinere Auschnitte entnommen. Dies liegt daran, dass das zufällige Entnehmen einer Stichprobe der gelben Taxidaten mehrere Minuten benötigt. 

Die Ausschnitte wurden aus den ML-Sichten erstellt, indem jeder Datensatz einen zufälligen neuen Hashwert bekam, nach welchem er sortiert ist. 

~\newline
Für Prognosen, die für ihr Training aggregierte Daten benötigen (z.B. Umsatzprognose), wurde ebenfalls eine Aggregats-Tabelle erstellt. 

Diese gruppiert die Machine-Learning-View nach Datum, Stunde, Startort,Zielort und aggregiert verschiedene Attribute. Dies führt zu einer Menge von ca. 1 Millionen Aggregat-Daten, hiervon wurden 10000 entfernt als Kontrollgruppe.