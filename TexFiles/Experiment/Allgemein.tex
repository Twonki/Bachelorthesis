\section{Ziele und Anforderungen}
\label{sec:Fallbeispiel} \label{sec:TaxiAllgemein} \label{sec:Allgemein}
Ziel des Fallbeispieles ist es, \textit{lohnenswerte} Prognosen anhand von realistischen Daten zu erheben und die Qualität der verwendeten Algorithmen objektiv zu bewerten. 

\paragraph{User-Stories} ~\newline
Als lohnenswert werden hierbei Fragestellungen bezeichnet, welche für ein Unternehmen einen Mehrwert darstellen. Konkret werden folgende User-Stories behandelt: ~\newline

\begin{itemize}
	\item Wie viele Taxis brauche ich kommenden Samstagmittag am Time-Square, wenn es sonnig wird?
	\item Wie viel Umsatz werde ich am ersten Oktoberwochenende machen?
	\item Zu welchem Ort wird eine Person an einem regnerischen Morgen aus Manhatten fahren?
	\item Am 23.12 um 01:00 endet die Weihnachtsfeier im Trump-Tower. Wieviele Passagiere wird das Taxi haben?
	\item Gibt es ein Muster, nach welchem mehr Trinkgeld gegeben wird?
	\item Wie viel Trinkgeld werden 3 Fahrgäste geben, wenn eine relativ kurze Strecke vom JFK-Airport gefahren wird?
	\item Wie lange wird ein Fahrgast brauchen, wenn er dem Taxi an der Freiheitsstatue sagt \textit{kurz zu warten}?
	\item Am 21.Juni um 14:30 stehen zwei Personen am Central Park bei Nebel. Werden Sie ein grünes oder ein gelbes Taxi nehmen? 
\end{itemize}

Es ist anzunehmen, das einige Prognosen deutlich bessere Ergebnisse liefern als andere. Dennoch sollen bewusst auch die Grenzen von Machine-Learning gezeigt werden.

~\newline Die vorgestellten User Stories werden in dieser Reihenfolge in den folgenden Abschnitten behandelt.

\paragraph{Anforderungen} ~\newline
Um eine Objektive Bewertung vorzunehmen, werden folgende Kriterien an die Durchführung der Experimente gestellt: 

\begin{itemize}
	\item \textbf{Harte Kriterien:} Die Tests liefern als Resultat eine Genauigkeit. 
	
	Eine Bewertung dieser Genauigkeit findet lediglich im Fazit statt. 
	\item \textbf{Wiederholbarkeit:} Eine Wiederholung der Tests muss dieselben Resultate liefern
	\item \textbf{Nachstellbarkeit:} Mithilfe dieses Experimentes muss der Leser im Stande sein, die gezeigten Ergebnisse selbst nachstellen zu können
\end{itemize}