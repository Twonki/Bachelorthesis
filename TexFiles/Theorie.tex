\chapter{Grundlagen zu Machine-Learning}
\label{cha:MachineLearning}
\label{cha:Theorie}
In diesem Kapitel werden die theoretischen Grundlagen ausgewählter Machine-Learning Algorithmen aus dem Bereich der Regression und Klassifikation vorgestellt. 

Es werden lediglich Algorithmen behandelt, die zum Feld des \textit{Supervised Learning} gezählt werden. Diese benötigen Trainingsdaten bestehend aus Ein- und Ausgabewerten, um das Modell daran auszurichten. Ein übliches Beispiel ist die Handschrifterkennung. Das Training wird im Abschnitt \ref{sec:NN} genauer Vorgestellt. 

~\newline Motivation für alle Algorithmen stellt die Annahme dar, das es innerhalb der vorliegenden Daten einen Zusammenhang der Werte gibt, eine Funktion welche einen Satz Daten auf ein Ergebnis abbildet. 

Falls eine Funktion existiert, ist diese allerdings häufig zu komplex, um von einem Menschen direkt formuliert zu werden.   

~\newline Ziel jeder der vorgestellten Algorithmen ist es, ein Modell zu erzeugen, welches möglichst genau die oben vermutete Funktion abschätzt. Hierfür wird eine (große) Menge an Trainingsdaten, sowie eine Menge an Kontrolldaten benötigt.   

~\newline Allgemeine Umsetzungen dieser Algorithmen finden sich im Abschnitt \ref{sec:R} zu R sowie konkret anhand des Fallbeispiels in Kapitel \ref{cha:Taxis}. 

Zunächst werden allerdings einige gemeinsame Begriffe erläutert. 
\section{Bias}
In diese Abschnitt werden kurz verschiedene Formen von \textit{Bias} (dt. Abweichung, Verzerrung) vorgestellt. Diese Abweichungen spielen in allen Formen des Machine-Learnings und in der Auswahl der Trainingsdaten eine wichtige Rolle (vgl. \cite{BiasTypes} Absatz 1 ) und werden in den entsprechenden Algorithmen berücksichtigt. Die nachfolgenden Arten von Bias stellen Überbegriffe dar - v.A. im Bereich der Psychologie wird deutlich genauer unterschieden.

\paragraph{Natürliche Varianz} Je nach Art und Gestalt der Erhebung können systematische Schwankungen der Werte auftreten. Diese stellen natürliche Verhältnisse dar, da kein perfektes Modell erfasst werden kann. 

Als Beispiel sei die Messung der Zimmertemperatur genannt: Zwei Thermometer können im selben Raum unterschiedliche Ergebnisse liefern - etwa weil sie auf unterschiedlichen Höhen befestigt sind oder eines im Windzug liegt. Es ist im Allgemeinen nicht möglich, ein perfektes Modell zu erstellen welches alle Faktoren berücksichtigt.

Die Natürliche Varianz ist als Hauptgrund zu nennen, warum in jedem (modernen) Machine-Learning Algorithmus eine Abweichung berücksichtigt ist.

~\newline Insbesondere ist  zu betonen, das die Genauigkeit eines Models, welches auf Machine-Learning beruht, nie höher sein kann als die Varianz der zugrunde liegenden Trainings-Daten. 
\paragraph{Selection Bias} Unter der Selektionsverzerrung versteht man einen Fehler der Ergebnisse, welcher durch die Auswahl einer \textbf{nicht repräsentativen} Stichprobe entsteht (vgl. \cite{SelectionBias} Definition). Ein Beispiel einer Selektionsverzerrung tritt auf \footnote{Es handelt sich hierbei um eine Vermutung}, wenn Anhand der Umfragen auf einer Messe für vegane Ernährung die Ernährungsgewohnheiten aller Deutscher interpretiert wird. 

Im Gegensatz dazu wäre diese Stichprobe sehr wohl geeignet, die Ernährung deutscher Veganer zu beurteilen.  

\paragraph{Confirmation Bias} Unter dem \textit{Confirmation Bias} (dt. Bestätigungsfehler) versteht man mehrere psychologische Aspekte die zu einer Verzerrung der Ergebnisse durch den Prüfer führen (vgl. \cite{ConfirmationBias} S. 21 Absatz 5 und S. 22 Absatz 1f). Im Wesentlichen bezieht sich diese Abweichung darauf, das unbewusst Ergebnisse so interpretiert werden um bestehende Meinungen zu bestätigen. Dies wird hauptsächlich über zwei Mechanismen erreicht: Die Interpretation nicht-übereinstimmender Ergebnisse und Daten als Fehlerhaft, sowie eine überproportionale Gewichtung übereinstimmender Ergebnisse. Hierzu gehört ebenfalls die explizite Suche nach Ergebnissen welche eine Hypothese bestätigen, ohne dieselbe Sorgfalt der Gegenhypothese zukommen zu lassen. 

\section{Definitionen und Notationen}
\label{Sec:Definitionen}\label{Definitionen}\label{Defs}
Nach dem Bias werden nun weitere Begriffe vorgestellt.
\paragraph{Feature} Unter einer \textit{Eigenschaft} versteht man eine konkrete Ausprägung eines Merkmals der Eingabewerte des Models.  

Die Summe aller Ausprägungen einer Eigenschaft bezeichnet man als Eigenschaftsvektor.  
\paragraph{Label \& Class} Als Label wird eine bestimmte Eigenschaft deklariert, welche v.A. dadurch definiert ist, das sie die Ausgabe des Models darstellt. 

Label und Klassen sind synonyme Bezeichner.
\paragraph{Accuracy} Die \textit{Genauigkeit} stellt ein Maß dafür dar, wie genau das Modell die Funktion darstellt. 

Je nach Art des Modells muss die Genauigkeit unterschiedlich evaluiert werden. 
\paragraph{Supervised Learning} Beim \textit{überwachtem Lernen} werden dem Algorithmus Trainingspaare aus Ein- und Ausgabewerten geliefert. 

Beispiele hierfür sind typische Klassifikationsverfahren, z.B. Handschrifterkennung.
\paragraph{Unsupervised Learning} Beim \textit{unüberwachtem Lernen} erhält der Algorithmus lediglich Eingabe-Werte und erzeugt selbst (Klassen von) Ausgabewerten. 

Ein Beispiel hierfür ist das \textit{Clustering}. 
\section{Lineare Regression}
\label{sec:LineareRegression}
Hier im Wesentlichen Stroetmann, das ist denke ich das beste. 
Was ist das und was macht es, warum ist das erstes Kapitel
\subsection{Konzept und Ziele linearer Regression}
Wofür brauche ich das, was kann ich damit machen, was kann ich damit nicht machen?
\paragraph{Beispiel}
z.B. Beispiel mit Gerade durch Punkte ziehen, Beispiel sollte für einfache und allgemeine Lineare Regression brauchbar sein 

Tabelle aus Werten, damit man später Funktion plotten kann und mehr Ressourcen hat 

\subsection{Einfache Lineare Regression}
Hier ist Lineare Regression auf einzelne Werte also 

$R^1 -> R^1$

\subsection{Allgemeine Lineare Regression}
Hier ist die komplizierte Regression gemeint, wie wir sie brauchen also

$R^n -> R^m$

mit vielen Vektoren, Matrizen und tollen Dingen

\subsection{Bewertung der Linearen Regression}
Wie berechne ich die statistische Signifikanz meines Linearen Modells?
\section{Klassifizerung}
\label{sec:Klassifizierung}
Hier vielleicht auch Stroetmann, oder etwas leichtgewichtigeres?
\subsection{Konzept und Ziele von Klassifizierung}
Hier Beispiel bringen, vllt Binäre Klassifizierung

\subsection{Definitionen und Notationen}
\paragraph{Features}
\paragraph{Label \& Klassen}
\paragraph{Model}
\paragraph{Accuracy}

\paragraph{Supervised Learning}
\paragraph{Unsupervised Learning}

\subsection{Digression: Gradientenanstieg}
Erklärung was Stochastic Gradient Ascent ist, kurzes Vorgreifen warum man es braucht
\subsection{Logistische Regression}

\subsubsection{Aktivierungsfunktion}
Hier wird kurz erklärt was es für Aktivierungsfunktionen gibt, der Bezug zur Stochastic/Wahrscheinlichkeit und kurzes Vorgreifen warum man es braucht
\paragraph{Sigmoid}
Was ist das, was macht die, Eigenschaften beim Ableiten (Siehe Stroetmann)

Bild zur Sigmoidfunktion als Plot, ArcTang und Gauss daneben

Kurze Erklärung warum man nicht die anderen benutzt
\paragraph{ReLU}
Was ist das?? Wieso ist das "So super gut"
\subsubsection{Modell der Logistischen Regression}
Wie spielen Aktivierungsfunktion, Optimierung und Lernen innerhalb der Logistischen Regression zusammen bei der Klassifizierung

\newpage
\section{Neuronale Netzwerke}
\label{sec:NN}
Dieser Abschnitt widmet sich den Konzepten künstlicher neuronaler Netze. Als Hauptquelle dient der Artikel \textit{Building a neural network from scratch} von David Selby \cite{SelbyNN} sowie die Vorlesung \textit{Artificial Intelligence} von Dr. Stroetmann \cite{stroetmann}.

~\newline Neuronale Netze erhielten ihren Namen, da man zu Beginn der Forschung dachte, dass das menschliche Gehirn wie ein \textit{computational graph} funktionierte. Diese These wurde biologisch weitgehend widerlegt und deswegen werden die neuronalen Netze der Informatik mit dem Zusatz \textit{künstlich} markiert. 

~\newline Im Folgenden wird zunächst der Aufbau des Modells und anschießend das Training vorgestellt.
\subsection{Modell künstlicher neuronaler Netze}
\begin{figure}[h]
	\begin{center}
		\includegraphics[width=0.4\linewidth]{Bilder/singleNeuron}
		\caption[Einzelnes Neuron: \url{
			http://caisplusplus.usc.edu/blog/curriculum/lesson4}]{Einzelnes Neuron}
		\label{fig:Neuron}
	\end{center}
\end{figure}

Grundidee des Modells bildet das Konzept eines \textbf{Neurons}. Dieses erhält Eingabewerte, und sobald ein gewisser Schwellwert erreicht wurde, \textit{feuert} es sein Signal ab, um andere Neuronen zu kontaktieren oder Handlungen hervorzurufen. 

Im Rahmen der Informatik äußert sich diese Neuronen-Logik durch eine Aktivierungsfunktion, die üblicherweise ohne Bedingung eine Ausgabe erzeugt. Dies zeigt Abbildung \ref{fig:Neuron}.
\begin{figure}[h]
\begin{center}
	\includegraphics[width=0.9\linewidth]{Bilder/petry19}
	\caption[Modell eines neuronalen Netzwerkes: \url{
		http://www.jurpc.de/jurpc/show?id=19990187}]{Modell eines neuronalen Netzwerkes}
	\label{fig:NN-Modell}
\end{center}
\end{figure}
~\newline Diese Neuronen werden zu einen Graphen, und sind in \textit{Schichten} (engl. Layer) angeordnet. 

Jedes Neuron einer Schicht erhält Eingaben von jedem Neuron der vorhergehenden\footnote{In diesem Fall spricht man von einem \textbf{vollvermaschten} neuronalen Netz. Es ist möglich, andere Formen der Verknüpfung oder Filterkriterien einzustellen.}. Diese Eingaben werden zusätzlich gewichtet. 

~\newline Die Eingabe in das neuronale Netzwerk erfolgt über den Inputlayer, welcher keine Aktivierungsfunktion hat. Die Ausgabe des neuronalen Netzwerkes erfolgt in der sog. Ausgabeschicht, welche je nach Art des Netzes einen (für Regressionen), zwei (für binäre Klassifikationen) oder $n$ (für n-Klassen Multiklassifikation) Knoten besitzt.

Die Neuronen zur Berechnung befinden sich in den \textit{Hidden Layers}. Gibt es mehr als einen Hidden-Layer spricht man von einem \textit{deep neuronal network}\footnote{Grund hierfür ist die Funktion der tieferen Schichten - Anstatt nur die Eingabe zu gewichten, werden hier weitere Features erkannt bzw. erzeugt, welche nur für den Algorithmus erkennbar sind.}. Dieses vollständige Modell zeigt Abbildung \ref{fig:NN-Modell}.

~\newline Um das Modell zu trainieren, muss ebenfalls der Fehler der Schätzung minimiert werden. Hierbei werden die Konzepte der linearen und logistischen Regression verwendet, jedoch mit dem Zusatz, das anstatt eines einzelnen Gewichtsvektors eine Gewichtsmatrix angepasst werden muss. 
\subsection{Forward Propagation}
Unter der \textit{Vorwärtsausbreitung} versteht man den Algorithmus, welcher einen Eingabevektor durch alle Gewichtsvektoren und Schichten transfomiert. 

Dieser \textit{Feed Forward}-Prozess kann sowohl iterativ über alle Vektoren erfolgen, oder zusammengefasst als Matrizenoperation.
\subsection{Backward Propagation}
Die \textit{Rückwärtsausrichtung} bezeichnet den Algorithmus, welcher die Gewichte anhand des gemessenen Fehlers nachjustiert. 

Hierbei wird in gleichem Maße wie in der logistischen Regression vorgegangen, mit dem Unterschied, dass die Gewichte in einer Matrix vorliegen. 

Die Eigenschaften der Aktivierungsfunktionen bezüglich ihrer Ableitung finden hier im besonderen Maße Anwendung, denn um von der Ausgabeschicht auf den letzten Hidden-Layer nachzujustieren, benötigt man die erste Ableitung. Um auf den nächsten Hidden Layer Einfluss zu nehmen, muss die zweite Ableitung gebildet werden (usw.). Eine mathematische Ausarbeitung findet sich unter \cite{colah} \textit{Computational Victories}.
\subsection{Training}
Das Training bezeichnet den (iterativen) Prozess, mit den vorliegenden Trainingsdaten zunächst Forward-Propagation durchzuführen, um anschließend mittels Backward Propagation das Netz auszurichten. 

~\newline Der Umfang dieses Trainings bleibt dem Anwender überlassen. Es ist möglich, bessere Ergebnisse zu erzielen, indem man mit denselben Daten häufiger trainiert. Einen solchen wiederholten Trainingsdurchlauf nennt man eine \textbf{Epoche}. 
\subsection{Bewertung des neuronalen Netzes}
Die Bewertung des neuronalen Netzes erfolgt, je nach Art des Ergebnisses, analog wie die der linearen oder logistischen Regression. 
\subsection{Einflüsse auf den Trainingserfolg}
Zum Abschluss dieses Abschnittes werden zusammenfassend die \textit{Stellschrauben} vorgestellt, anhand derer Änderungen des Trainingserfolges erzielt werden können. 

\begin{itemize}
	\item \textbf{Netzaufbau und Struktur:} Die Anzahl der Knoten, Schichten, und Einstellungen zur Verknüpfung können variiert werden.
	\item \textbf{Optimierungsfunktion:} Hierbei kann der grundlegende Algorithmus (Gradient Descent oder Stochastic Gradient Descent), sowie Trainingsparameter (Lerngeschwindigkeit, Beschleunigung, Verfall) gewählt werden.
	\item \textbf{Aktivierungsfunktion der Neuronen}
	\item \textbf{Menge der Trainingsdaten:} Eine größere Menge an Trainingsdaten hilft maßgeblich, den Sachverhalt besser erfassen zu können. Auch sind große Datenmengen notwendig, um bei komplexeren Netzen \textit{overfitting} zu vermeiden.
	\item \textbf{Anzahl der Features} 
	\item \textbf{Anzahl der Epochen und Iterationen}
	
\end{itemize}

Konkrete Anwendungen dieser Parameter und die damit erzielten Ergebnisse finden sich im Kapitel \ref{cha:Experiment} dieser Arbeit.