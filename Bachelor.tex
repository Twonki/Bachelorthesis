% Präambel
\documentclass[12pt,a4paper,oneside, 
liststotoc, 					% Tabellen- und Abbildungsverzeichnis ins Inhaltsverzeichnis
bibtotoc,						% Literaturverzeichnis ins Inhaltsverzeichnis aufnehmen
titlepage, 						% Titlepage-Umgebung statt \maketitle
headsepline, 					% horizontale Linie unter Kolumnentitel
%abstracton,					% Überschrift beim Abstract einschalten, Abstract muss dazu in {abstract}-Umgebung stehen
%DIV11,							% auskommentieren, um den Seitenspiegel zu vergrößern
BCOR6mm,						% Bindekorrektur, die den Seitenspiegel um 6mm nach rechts verschiebt,
]{scrreprt}		

% Laden verschiedener Packages
\include{Packages}
%Einstellungen, Farben, Commands, 
\newcommandx{\unsure}[2][1=]{\todo[linecolor=red,backgroundcolor=red!25,bordercolor=red,#1]{#2}}
\newcommandx{\change}[2][1=]{\todo[linecolor=blue,backgroundcolor=blue!25,bordercolor=blue,#1]{#2}}
\newcommandx{\info}[2][1=]{\todo[linecolor=OliveGreen,backgroundcolor=OliveGreen!25,bordercolor=OliveGreen,#1]{#2}}
\newcommandx{\improvement}[2][1=]{\todo[linecolor=Plum,backgroundcolor=Plum!25,bordercolor=Plum,#1]{#2}}
\newcommandx{\thiswillnotshow}[2][1=]{\todo[disable,#1]{#2}}

\newtheorem{satz}{Satz}
\newtheorem{formel}{Formel}
\newtheorem{defi}{Definition}[section]

\pgfkeys{
	/kiviatgrad/simplify label/.code={
		\ifx\nv\undefined\else
		\pgfmathparse{Mod(\nv,5)}
		\ifdim\pgfmathresult pt>0pt
		\tikzset{opacity=0}
		\fi
		\fi
	}
}	
% ------------------Setting up environment for codes ---------------
\lstset{
	numbers=left,
	language=python,  
	morekeywords = {FROM, RUN, ADD, WORKDIR, EXPOSE, ENTRYPOINT, USER},
	numberstyle=\small, 
	numbersep=8pt,  
	frame=bt,
	keywordstyle=\color{ForestGreen}\bfseries,
	commentstyle=\color{cyan},
	framexleftmargin=0pt
}
\definecolor{groovyblue}{HTML}{0000A0}
\definecolor{groovygreen}{HTML}{008000}
\definecolor{darkgray}{rgb}{.4,.4,.4}

\lstset{language=C,
	numbers=left,
	language=python,  
%	morekeywords = {FROM, RUN, ADD, WORKDIR, EXPOSE, ENTRYPOINT, USER},
	numberstyle=\small, 
	numbersep=8pt,  
	frame=bt,
%	keywordstyle=\color{ForestGreen}\bfseries,
%	commentstyle=\color{cyan},
	framexleftmargin=0pt
}
\lstdefinestyle{BasicBashStyle}{
	language=C,
	basicstyle=\footnotesize\sffamily,
	numbers=left,
	numberstyle=\tiny,
	numbersep=3pt,
	frame=tb,
	columns=fullflexible,
	backgroundcolor=\color[rgb]{0.910,0.933,0.970},
	linewidth=0.95\linewidth,
	xleftmargin=0.1\linewidth
}
\usetikzlibrary{arrows} % define style of tkiz kiviat
\renewcommand{\lstlistingname}{Code}

% Abkürzungen
\newcommand{\ua}{\mbox{u.\,a.\ }}
\newcommand{\zB}{\mbox{z.\,B.\ }}
\newcommand{\bs}{$\backslash$}

\renewcommand{\nomname}{Abkürzungsverzeichnis}

% -------------------------------------------------------------------------------------------
% Definition der Kopf- und Fußzeilen
\lhead{}								% Kopf links
\chead{}								% Kopf mitte
\rhead{\sffamily{\leftmark}}				% Kopf rechts
\lfoot{}								% Fuß links
\cfoot{\sffamily{\thepage}}				% Fuß mitte
\rfoot{\sffamily{\autor}}				% Fuß rechts
\renewcommand{\headrulewidth}{0.4pt}	% Liniendicke Kopf
\renewcommand{\footrulewidth}{0.4pt}	% Liniendicke Fuß


\makenomenclature							% Abkürzungsverzeichnis erstellen
%\input{Inhalt/abkuerzungen}					% Datei mit Abkürzungen laden

% Definition of colors
\definecolor{lightblue}{rgb}{0.910,0.933,0.970}
\definecolor{lightred}{RGB}{247,238,232}
\definecolor{monochromeLightblue}{RGB}{165,188,222}
\definecolor{monochromeLightred}{RGB}{222,188,165}
\definecolor{kiviatOne}{RGB}{137,193,30}
\definecolor{kiviatTwo}{RGB}{20,128,120}
\definecolor{kiviatThree}{RGB}{208,108,32}
\definecolor{kiviatFour}{RGB}{168,26,104}

\newcommand\ColorBox[1]{\textcolor{#1}{\rule{2.5ex}{2.5ex}}}

% ----------------------------------- Links Styling -----------------------------------------
\hypersetup{
	%pdfborder = {0 0 0},
	bookmarks=true,         % show bookmarks bar?
	unicode=false,          % non-Latin characters in Acrobat’s bookmarks
	pdftitle={Analyse von Graphalgorithmen}
	pdfauthor={Leonhard Applis},     % author
	pdfsubject={Bachelorthesis},   % subject of the document
	pdfcreator={Leonhard Applis},   % creator of the document
	pdfproducer={Leonhard Applis}, % producer of the document
	colorlinks=true,       % false: boxed links; true: colored links
	linkcolor=blue,          % color of internal links (change box color with inkbordercolor)
	citecolor=CadetBlue,        % color of links to bibliography
	filecolor=magenta,      % color of file links
	urlcolor=cyan,           % color of external links
}

% -------------------------------------------------------------------------------------------
%                     Persönliche Daten
% -------------------------------------------------------------------------------------------


\newcommand{\titel}{Machine Learning und Prognosen}
\newcommand{\shellcmd}[1]{\\\indent\indent\texttt{\footnotesize\# #1}\\}
\newcommand{\untertitel}{Umsetzung mit R im SQLServer 2017 anhand von Taxifahrten in New York}
\newcommand{\arbeit}{Bachelorthesis}
\newcommand{\studiengang}{Angewandte Informatik}
\newcommand{\autor}{Leonhard Applis}
\newcommand{\matrikelnr}{2086307}
\newcommand{\kurs}{TINF15AIBI}
\newcommand{\firma}{Atos Information Technology GmbH}
\newcommand{\betreuerfirma}{Jonas Mauer}
\newcommand{\abgabe}{24.09.2018}
\newcommand{\betreuerdhbw}{Prof. Dr. Rainer Colgen}
\newcommand{\jahr}{2018}			% für Angabe im Copyright-Vermerk der Titelseite


% -------------------------------------------------------------------------------------------
%                     Beginn des Dokumenteninhalts
% -------------------------------------------------------------------------------------------


\begin{document}
\setcounter{secnumdepth}{3}					% Nummerierungstiefe fürs Inhaltsverzeichnis
\setcounter{tocdepth}{3}
\sffamily									% für die Titelei serifenlose Schrift verwenden

% ------------------------------ Titelei -----------------------------------------------------

\thispagestyle{plain}
\begin{titlepage}
\enlargethispage{3.5cm}
\sffamily 								% Serifenlose Grundschrift für die Titelseite einstellen
\begin{minipage}{\textwidth}
	\vspace{-2cm}
	\noindent \includegraphics[scale=0.5]{Bilder/logo_atos.png} \hfill  \includegraphics[scale=1.0]{Bilder/logo_dhbw.jpg}\\[5ex]
\end{minipage} 
\begin{center}

\huge{\textsc{\textbf{\titel}}}\\[1.5ex]
\Large{\textbf{\untertitel}}\\[5ex]
\LARGE{\textbf{\arbeit}}\\[2ex]
\normalsize{~}\\[3ex]
\Large{Studiengang \textit{\studiengang}}\\[1ex]
\normalsize{Duale Hochschule Baden-Württemberg Mannheim}\\[5ex]
von\\[1ex] \autor \\[12ex]
\end{center}

\begin{flushleft}

\begin{tabular}{ll}
Abgabedatum:					& \quad \abgabe \\ 
Matrikelnummer, Kurs: 			& \quad \matrikelnr , \kurs \\ 
Ausbildungsfirma:	 			& \quad \firma \\
Betreuer der Ausbildungsfirma   & \quad \betreuerfirma \\
Gutachter der Dualen Hochschule: & \quad \betreuerdhbw \\ 
[6ex]%formerly 5ex

\end{tabular} 
\end{flushleft}
\end{titlepage}
%Starting Correct Spacing here...
\onehalfspacing 				% erzeugt die Titelseite
\pagenumbering{Roman}						% große, römische Seitenzahlen für Titelei
\addchap*{Eidesstattliche Erklärung}
Ich versichere hiermit, dass ich meine \arbeit~ mit dem Thema
\begin{quote}
\textit{\titel} \textit{\untertitel }
\end{quote}
selbständig verfasst und keine anderen als die angegebenen Quellen und Hilfsmittel benutzt habe. Die Arbeit wurde bisher keiner anderen Prüfungsbehörde vorgelegt und auch nicht veröffentlicht.


Ich versichere zudem, dass die eingereichte elektronische Fassung mit der gedruckten Fassung übereinstimmt.\\[10ex]

Fürth, den \today \\[4ex]


\rule[-0.2cm]{5cm}{0.5pt} \\

\textsc{\autor} \\[10ex]
 				% Einbinden der eidestattlichen Erklärung
\chapter*{Abstract} %*-Variante sorgt dafür, das Abstract nicht im Inhaltsverzeichnis auftaucht
Englisch Abstract to be done
~\newline
~\newline
\begin{flushleft}
	\begin{tabular}{ll}
		\textbf{title:} &\quad Machine-Learning and Prognosis \\
		\textbf{author:}  &\quad Leonhard Applis \\
		\textbf{matriculation number:} &\quad 2086307 \\
		\textbf{class:} &\quad TINF15/AI-BI \\
		\textbf{supervisor DHBW:} &\quad ??? \\
		\textbf{supervisor Atos:} & \quad \betreuerfirma \\
		[6ex]%formerly 5ex
	\end{tabular} 
\end{flushleft}


\chapter*{Kurzfassung} 
Deutscher Abstract muss gemacht werden
~\newline
~\newline
\begin{flushleft}
	\begin{tabular}{ll}
		Titel:& \quad \titel \\ 
		Author:& \quad Leonhard Applis \\
		Matrikelnummer: & \quad \matrikelnr  \\
		Kurs: & \quad \kurs \\ 
		Betreuer der Dualen Hochschule: & \quad \betreuerdhbw \\ 
		Betreuer der Firma: & \quad \betreuerfirma \\
		[6ex]%formerly 5ex	
	\end{tabular} 
\end{flushleft}   				% Einbinden des Abstracts

\tableofcontents							% Erzeugen des Inhalsverzeichnisses
\printnomenclature[2.0cm]					% Erzeugen des Abkürzungsverzeichnisses
\listoffigures 								% Erzeugen des Abbildungsverzeichnisses 
%\listoftables 								% Erzeugen des Tabellenverzeichnisses
\pagebreak
% ------------------ Graphic Extension ------------------------------------------------------

% --------------------------------------------------------------------------------------------
%                    Inhalt der Bachelorarbeit
%---------------------------------------------------------------------------------------------
\pagenumbering{arabic}						% arabische Seitenzahlen für den Hauptteil
\pagestyle{fancy}					
\rmfamily

%\include{TexFiles/Abkuerzungen}
\chapter{Einleitung}
\label{cha:Einleitung}
\setlength{\epigraphwidth}{4in}
\epigraph{“Hasta la Vista, Baby!"}{\textit{Arnold Schwarzenegger} \textup{ in Terminator 2}}

Dieses Zitat zählt wohl zu den bekanntesten der Filmgeschichte, und markiert einen der ersten bühnenreifen Auftritte \textit{künstlicher Intelligenz}. Neben österreichischen Bodybuildern beschäftigt dieses Thema seit bald einem Jahrhundert Wissenschaftler, Ethiker und Science-Fiction-Fans gleichermaßen. Was vor zwei Jahrzehnten noch genauso fantasievoll wie schwebende Autos klang, wird in den Softwareschmieden des 21. Jahrhunderts Wirklichkeit: 

Künstliche Intelligenzen besiegen Schachprofis, organisieren unsere Kalender, analysieren Bilder und helfen Pandemien einzudämmen. Neben diesen bahnbrechenden Erfolgen gibt es auch weiterhin vielversprechende Forschung in diesem Themengebiet, zum Beispiel computergesteuerte Autos. Aber was ist künstliche Intelligenz eigentlich?

~\newline Der Begriff der künstlichen Intelligenz ist sehr weit gefächert - ein Kernelement davon stellt das \textit{Machine Learning} dar. Dieser Bereich, der sich auf die Erstellung von Modellen anhand von Trainingsdaten stützt, hat durch \textit{neuronale Netze} stark an Bedeutung gewonnen. Die Gründe hierfür sind vielseitig, dennoch sind zwei im Besonderen zu nennen: Zum Einen sind Computer deutlich leistungsfähiger geworden, und Aufgaben die früher einen Supercomputer benötigten, sind heute durch ein Smartphone umsetzbar. Zum Anderen sind deutlich mehr Bereiche digitalisiert, und die gewonnenen Daten detaillierter. 

~\newline Genau diesem Themengebiet widmet sich diese Bachelorarbeit: Machine-Learning und explizit neuronalen Netzen. 
\section{Ziel der Arbeit}
\label{sec:ZielDerArbeit}
Die Ziele dieser Arbeit sind es, ein Grundverständnis für Machine-Learning Algorithmen zu schaffen, und einen Überblick wie diese mit dem SQL-Server 2017 umgesetzt werden. 

Hierfür wird die Theorie verschiedener Algorithmen detailliert vorgestellt und in der Programmiersprache R umgesetzt. 

Ebenfalls wird ein detailliertes Fallbeispiel mit Versuchsaufbau und Ergebnissen erarbeitet, damit der Leser eine Einschätzung der Algorithmen vornehmen kann ohne selbst Experimente durchzuführen.

~\newline Es ist \textbf{nicht} Ziel dieser Arbeit, einen Vergleich zwischen unterschiedlichen Machine-Learning Ansätzen und Frameworks zu ziehen. Auch wird ausschließlich mit R und dem SQL-Server gearbeitet. 

Zudem werden weder Grundlagen der Sprachen SQL und R, noch die Vorbereitung des Fallbeispiels geschildert.

\section{Aufbau der Arbeit}
Kapitel \ref{cha:Theorie} dieser Arbeit behandelt die Theorie zu modernen Ansätzen des Machine-Learnings. Es werden die Algorithmen für lineare Regression, logistische Regression sowie neuronale Netzwerke detailliert vorgestellt (in Reihenfolge der Nennung). Dieses Kapitel stellt einen rein theoretischen Teil der Arbeit dar, und beinhaltet keine Umsetzung der Algorithmen als Programme.

~\newline Darauf aufbauend  werden in Kapitel \ref{cha:SQLServer} zunächst Grundlagen zu Microsofts SQL-Server 2017 und R geklärt, anschließend liegt der Schwerpunkt des Kapitels auf der Umsetzung von Machine-Learning Algorithmen in R. Innerhalb des Abschnittes \ref{sec:MLSQL} finden sich Codebeispiele zu Prognosemodellen in T-SQL und R.

~\newline Kapitel \ref{cha:Taxis} widmet sich der Umsetzung des Fallbeispiels eines Taxiunternehmens. Zunächst werden in Abschnitt \ref{sec:TaxiAllgemein} die Ausgangslage der Daten sowie die Ziele des Fallbeispiels exakt definiert. 

In Abschnitt \ref{sec:Daten} werden die Stammdaten des Taxiunternehmens und die Wetterdaten in Eigenschaften, Umfang und Bedeutung für Machine Learning dargestellt. 

~\newline Hauptteil des Kapitels \ref{cha:Taxis} bilden die Unterabschnitte \ref{sec:TipPred} bis \ref{sec:PasPred}, welche das Definieren, Training und Bewertung verschiedener neuronaler Netze vorstellen. Anschließend werden in Abschnitt \ref{sec:BestPractices} die Best Practices im Umgang mit neuronalen Netzen im SQL-Server vorgestellt.

~\newline Abschluss der Arbeit bildet in Kapitel \ref{cha:Fazit} ein Fazit über die Qualität der Prognosen unter Berücksichtigung der Komplexität einzelner Teilaufgaben. 
\section{Voraussetzungen an den Leser}
\label{sec:Vorraussetzungen}
Innerhalb dieses Punktes werden die Kenntnisse abgesteckt, die der Leser für das Verständnis der Arbeit benötigt, welche \textbf{nicht} im Rahmen dieser Arbeit vorgestellt werden. 

\begin{itemize}
	\item \textbf{Lineare Algebra:} Im Rahmen dieser Arbeit werden komplexe Algorithmen und Konzepte der mehrdimensionalen Algebra benötigt. 
	
	Schwerpunkte liegen hier v.A. auf dem Lösen von mehrdimensionalen Gleichungen und Matrixoperationen.
	\item \textbf{Stochastik:}  Zur Bewertung der Algorithmen werden tiefere Kenntnisse der Stochastik und Statistik benötigt. Die benötigten Schwerpunktthemen sind Verteilungsfunktionen, Hypothesentests und Korrelation.
	\item \textbf{R:} Die Programmiersprache R muss dem Leser im Umfang eines Basiskurses bekannt sein. Sie wird im Zuge der Arbeit verwendet, allerdings werden grundlegende Elemente nicht vorgestellt. 
	\item \textbf{SQL:} Die Konzepte von SQL und der Dialekt von T-SQL sind in fortgeschrittenen Zügen benötigt. Die Verwendung des R-Codes innerhalb eines SQL-Servers wird im Zuge der Arbeit vorgestellt. 
\end{itemize}

\chapter{Grundlagen zu Machine-Learning}
\label{cha:MachineLearning}
\label{cha:Theorie}
Hier gebe ich ein Vorwort, wie heftig der Spaß hier wird und warum ich zuerst mit der Theorie Anfange. 
\section{Lineare Regression}
\label{sec:LineareRegression}
Hier im Wesentlichen Stroetmann, das ist denke ich das beste. 
Was ist das und was macht es, warum ist das erstes Kapitel
\subsection{Konzept und Ziele linearer Regression}
Wofür brauche ich das, was kann ich damit machen, was kann ich damit nicht machen?
\paragraph{Beispiel}
z.B. Beispiel mit Gerade durch Punkte ziehen, Beispiel sollte für einfache und allgemeine Lineare Regression brauchbar sein 

Tabelle aus Werten, damit man später Funktion plotten kann und mehr Ressourcen hat 

\subsection{Einfache Lineare Regression}
Hier ist Lineare Regression auf einzelne Werte also 

$R^1 -> R^1$

\subsection{Allgemeine Lineare Regression}
Hier ist die komplizierte Regression gemeint, wie wir sie brauchen also

$R^n -> R^m$

mit vielen Vektoren, Matrizen und tollen Dingen

\subsection{Bewertung der Linearen Regression}
Wie berechne ich die statistische Signifikanz meines Linearen Modells?
\section{Klassifizerung}
\label{sec:Klassifizierung}
Hier vielleicht auch Stroetmann, oder etwas leichtgewichtigeres?
\subsection{Konzept und Ziele von Klassifizierung}
Hier Beispiel bringen, vllt Binäre Klassifizierung

\subsection{Definitionen und Notationen}
\paragraph{Features}
\paragraph{Label \& Klassen}
\paragraph{Model}
\paragraph{Accuracy}

\paragraph{Supervised Learning}
\paragraph{Unsupervised Learning}

\subsection{Digression: Gradientenanstieg}
Erklärung was Stochastic Gradient Ascent ist, kurzes Vorgreifen warum man es braucht
\subsection{Logistische Regression}

\subsubsection{Aktivierungsfunktion}
Hier wird kurz erklärt was es für Aktivierungsfunktionen gibt, der Bezug zur Stochastic/Wahrscheinlichkeit und kurzes Vorgreifen warum man es braucht
\paragraph{Sigmoid}
Was ist das, was macht die, Eigenschaften beim Ableiten (Siehe Stroetmann)

Bild zur Sigmoidfunktion als Plot, ArcTang und Gauss daneben

Kurze Erklärung warum man nicht die anderen benutzt
\paragraph{ReLU}
Was ist das?? Wieso ist das "So super gut"
\subsubsection{Modell der Logistischen Regression}
Wie spielen Aktivierungsfunktion, Optimierung und Lernen innerhalb der Logistischen Regression zusammen bei der Klassifizierung

\newpage
\section{Neuronale Netzwerke}
\label{sec:NN}
Dieser Abschnitt widmet sich den Konzepten künstlicher neuronaler Netze. Als Hauptquelle dient der Artikel \textit{Building a neural network from scratch} von David Selby \cite{SelbyNN} sowie die Vorlesung \textit{Artificial Intelligence} von Dr. Stroetmann \cite{stroetmann}.

~\newline Neuronale Netze erhielten ihren Namen, da man zu Beginn der Forschung dachte, dass das menschliche Gehirn wie ein \textit{computational graph} funktionierte. Diese These wurde biologisch weitgehend widerlegt und deswegen werden die neuronalen Netze der Informatik mit dem Zusatz \textit{künstlich} markiert. 

~\newline Im Folgenden wird zunächst der Aufbau des Modells und anschießend das Training vorgestellt.
\subsection{Modell künstlicher neuronaler Netze}
\begin{figure}[h]
	\begin{center}
		\includegraphics[width=0.4\linewidth]{Bilder/singleNeuron}
		\caption[Einzelnes Neuron: \url{
			http://caisplusplus.usc.edu/blog/curriculum/lesson4}]{Einzelnes Neuron}
		\label{fig:Neuron}
	\end{center}
\end{figure}

Grundidee des Modells bildet das Konzept eines \textbf{Neurons}. Dieses erhält Eingabewerte, und sobald ein gewisser Schwellwert erreicht wurde, \textit{feuert} es sein Signal ab, um andere Neuronen zu kontaktieren oder Handlungen hervorzurufen. 

Im Rahmen der Informatik äußert sich diese Neuronen-Logik durch eine Aktivierungsfunktion, die üblicherweise ohne Bedingung eine Ausgabe erzeugt. Dies zeigt Abbildung \ref{fig:Neuron}.
\begin{figure}[h]
\begin{center}
	\includegraphics[width=0.9\linewidth]{Bilder/petry19}
	\caption[Modell eines neuronalen Netzwerkes: \url{
		http://www.jurpc.de/jurpc/show?id=19990187}]{Modell eines neuronalen Netzwerkes}
	\label{fig:NN-Modell}
\end{center}
\end{figure}
~\newline Diese Neuronen werden zu einen Graphen, und sind in \textit{Schichten} (engl. Layer) angeordnet. 

Jedes Neuron einer Schicht erhält Eingaben von jedem Neuron der vorhergehenden\footnote{In diesem Fall spricht man von einem \textbf{vollvermaschten} neuronalen Netz. Es ist möglich, andere Formen der Verknüpfung oder Filterkriterien einzustellen.}. Diese Eingaben werden zusätzlich gewichtet. 

~\newline Die Eingabe in das neuronale Netzwerk erfolgt über den Inputlayer, welcher keine Aktivierungsfunktion hat. Die Ausgabe des neuronalen Netzwerkes erfolgt in der sog. Ausgabeschicht, welche je nach Art des Netzes einen (für Regressionen), zwei (für binäre Klassifikationen) oder $n$ (für n-Klassen Multiklassifikation) Knoten besitzt.

Die Neuronen zur Berechnung befinden sich in den \textit{Hidden Layers}. Gibt es mehr als einen Hidden-Layer spricht man von einem \textit{deep neuronal network}\footnote{Grund hierfür ist die Funktion der tieferen Schichten - Anstatt nur die Eingabe zu gewichten, werden hier weitere Features erkannt bzw. erzeugt, welche nur für den Algorithmus erkennbar sind.}. Dieses vollständige Modell zeigt Abbildung \ref{fig:NN-Modell}.

~\newline Um das Modell zu trainieren, muss ebenfalls der Fehler der Schätzung minimiert werden. Hierbei werden die Konzepte der linearen und logistischen Regression verwendet, jedoch mit dem Zusatz, das anstatt eines einzelnen Gewichtsvektors eine Gewichtsmatrix angepasst werden muss. 
\subsection{Forward Propagation}
Unter der \textit{Vorwärtsausbreitung} versteht man den Algorithmus, welcher einen Eingabevektor durch alle Gewichtsvektoren und Schichten transfomiert. 

Dieser \textit{Feed Forward}-Prozess kann sowohl iterativ über alle Vektoren erfolgen, oder zusammengefasst als Matrizenoperation.
\subsection{Backward Propagation}
Die \textit{Rückwärtsausrichtung} bezeichnet den Algorithmus, welcher die Gewichte anhand des gemessenen Fehlers nachjustiert. 

Hierbei wird in gleichem Maße wie in der logistischen Regression vorgegangen, mit dem Unterschied, dass die Gewichte in einer Matrix vorliegen. 

Die Eigenschaften der Aktivierungsfunktionen bezüglich ihrer Ableitung finden hier im besonderen Maße Anwendung, denn um von der Ausgabeschicht auf den letzten Hidden-Layer nachzujustieren, benötigt man die erste Ableitung. Um auf den nächsten Hidden Layer Einfluss zu nehmen, muss die zweite Ableitung gebildet werden (usw.). Eine mathematische Ausarbeitung findet sich unter \cite{colah} \textit{Computational Victories}.
\subsection{Training}
Das Training bezeichnet den (iterativen) Prozess, mit den vorliegenden Trainingsdaten zunächst Forward-Propagation durchzuführen, um anschließend mittels Backward Propagation das Netz auszurichten. 

~\newline Der Umfang dieses Trainings bleibt dem Anwender überlassen. Es ist möglich, bessere Ergebnisse zu erzielen, indem man mit denselben Daten häufiger trainiert. Einen solchen wiederholten Trainingsdurchlauf nennt man eine \textbf{Epoche}. 
\subsection{Bewertung des neuronalen Netzes}
Die Bewertung des neuronalen Netzes erfolgt, je nach Art des Ergebnisses, analog wie die der linearen oder logistischen Regression. 
\subsection{Einflüsse auf den Trainingserfolg}
Zum Abschluss dieses Abschnittes werden zusammenfassend die \textit{Stellschrauben} vorgestellt, anhand derer Änderungen des Trainingserfolges erzielt werden können. 

\begin{itemize}
	\item \textbf{Netzaufbau und Struktur:} Die Anzahl der Knoten, Schichten, und Einstellungen zur Verknüpfung können variiert werden.
	\item \textbf{Optimierungsfunktion:} Hierbei kann der grundlegende Algorithmus (Gradient Descent oder Stochastic Gradient Descent), sowie Trainingsparameter (Lerngeschwindigkeit, Beschleunigung, Verfall) gewählt werden.
	\item \textbf{Aktivierungsfunktion der Neuronen}
	\item \textbf{Menge der Trainingsdaten:} Eine größere Menge an Trainingsdaten hilft maßgeblich, den Sachverhalt besser erfassen zu können. Auch sind große Datenmengen notwendig, um bei komplexeren Netzen \textit{overfitting} zu vermeiden.
	\item \textbf{Anzahl der Features} 
	\item \textbf{Anzahl der Epochen und Iterationen}
	
\end{itemize}

Konkrete Anwendungen dieser Parameter und die damit erzielten Ergebnisse finden sich im Kapitel \ref{cha:Experiment} dieser Arbeit.
\chapter{SQLServer 2017 und R}
\label{cha:SQLServer}
\label{cha:R}
In diesem Kapitel werden zunächst die Umgebung des SQL-Servers 2017 sowie die Programmierprache R kurz vorgestellt, bevor in Abschnitt \ref{sec:MLSQL} eine konkrete Umsetzung der unter Kapitel \ref{cha:Theorie} gezeigten Algorithmen mit R erfolgt.
\chapter{SQLServer 2017 und R}
\label{cha:SQLServer}
\label{cha:R}
In diesem Kapitel werden zunächst die Umgebung des SQL-Servers 2017 sowie die Programmierprache R kurz vorgestellt, bevor in Abschnitt \ref{sec:MLSQL} eine konkrete Umsetzung der unter Kapitel \ref{cha:Theorie} gezeigten Algorithmen mit R erfolgt.
\chapter{SQLServer 2017 und R}
\label{cha:SQLServer}
\label{cha:R}
In diesem Kapitel werden zunächst die Umgebung des SQL-Servers 2017 sowie die Programmierprache R kurz vorgestellt, bevor in Abschnitt \ref{sec:MLSQL} eine konkrete Umsetzung der unter Kapitel \ref{cha:Theorie} gezeigten Algorithmen mit R erfolgt.
\input{Texfiles/SQLServer/SQLServer}
\input{Texfiles/SQLServer/R}
\input{Texfiles/SQLServer/MachineLearning}
\section{Programmiersprache R}
\label{sec:R}
Die Sprache R stellt vorallem für Statistiker und Psychologen ein Standard-Tool dar, dennoch ist sie auch bei Data-Scientisten beliebt für ihre vielseitigen Plot- und Modellierungs-Möglichkeiten. Im Rahmen dieser Arbeit wurden alle Modelle mithilfe von R erstellt, was innerhalb dieses Abschnittes vorgestellt wird:

~\newline R ist eine Sprache sowie eine Entwicklungsumgebung für statistische Berechnungen und Grafiken. R ist ein GNU-Projekt und beruht auf der Sprache $S$, welche von John Chambers et al. entwickelt wurde. R stellt eine Implementation von S dar (vgl. \cite{RProject} Absatz 1) und liegt als Open Source Projekt vor.

~\newline R bietet eine große Bandbreite an statistischen Funktionen (z.B. Lineare und Nichtlineare Regression, Klassifikation und Signifikanztests) sowie grafische Aufbereitungen dieser und ist hochgradig Modular (vgl. \cite{RProject} Absatz 2). 

~\newline Die größten Stärken von R liegen neben der einfachen Anwendung statistischer Funktionen in der Aufbereitung als Plots. R erzeugt schnell verständliche Grafiken der Daten, bieten erfahrenen Nutzern allerdings viele Optionen exakt benötigte Darstellungen zu erzeugen. 

\paragraph{ R Umfeld} 
R umfasst folgende integrierte Dienste (Siehe \cite{RProject} Absatz 5f):
\begin{enumerate}
	\item Eine Speichereinheit und Daten-Engine
	\item Eine Umgebung für Berechnungen auf Listen, Vektoren und insbesondere Matrizen
	\item Eine Sammlung an Werkzeugen zur Datenanalyse, statistischen Auswertung und Erzeugung von Grafiken
	\item Eine Programmiersprache, welche auf Bedingungen, Schleifen und nutzerdefinierte Funktionen eingeht
\end{enumerate}

Für rechen- und Zeitintensive Operationen kann zusätzlich C und C++ Code zur Laufzeit eingebunden werden. Ebenso kann man mit C direkt Objekte manipulieren. 

~\newline Die Funktionalitäten von R können über ein Paket-System erweitert werden. Das wichtigste Paket innerhalb dieser Arbeit stellt \textit{RevoScaleR} dar, welches eine persistente Speicherung von Datenobjekten in Datenbanken und Files ermöglicht. 

\paragraph{Besonderheiten in der Programmierung}~\newline
Die wichtigste Besonderheit in R ist, das jedes Objekt als Vektor aufgefasst wird. Ein einzelner Wert wird ebenfalls als Vektor der Größe 1 betrachtet.

~\newline Vektoren können für arithmetische Ausdrücke verwendet werden, in diesem Fall werden die Operationen Element für Element ausgeführt. Zwei Vektoren, welche in einer Anweisung vorkommen, müssen nicht die selbe Länge besitzen. Falls dies nicht der Fall ist, ist die Ausgabe der Anweisung ein Vektor der Länge des längsten Vektors. Die kürzen Vektoren werden solange wiederholt, bis sie die Länge des längsten Vektors erreicht haben.

Insbesondere konstanten werden auf jedes Element angewendet (vgl. \cite{RIntro} Seite 13 Abschnitt 2.2 \textit{Vektorarithmetik} Absatz 1).

~\newline Diese Eigenschaft der Vektoren ist vor dem Hintergrund, mit Datenbanken zu arbeiten ein zweischneidiges Schwert: Zum einen werden die Operationen und Anweisungen sehr \textit{einfach} und Übersichtlich (Hilfsstrukturen für Schleifen entfallen), allerdings bringt v.A. die Wiederholung der kleineren Vektoren erhebliche Fehlerquellen mit sich. 

~\newline Ein Faktor ist ein Vektor mit einem fest definierten Wertebereich (z.B. ein Charakter-Vektor, Siehe auch \cite{RIntro} Kapitel 4 \textit{Ordered and Unordered Factors} Absatz 1).

~\newline Ein \textit{Array} stellt in R eine Kombination aus einem Wert-Vektor und einem Dimensions-Vektor dar. Der Dimensionsvektor gibt hierbei eine Form für den Wert-Vektor dar, und bestimmt in welcher Reihenfolge und ggfs. mit welchen Eigenschaften Operationen ausgeführt werden. Für arithmetische Operationen zweier Arrays wird ebenfalls die o.G. \textit{Recycling Rule} angewendet. Im Falle einer Anweisung eines Arrays und eines Vektors, wird zunächst versucht aus dem Vektor ein Array derselben Dimension zu erzeugen. Eine Matrix stellt ein zweidimensionales Array dar. 

~\newline Ein \textit{Data-Frame} stellt eine besondere Form einer Liste dar, die folgende Eigenschaften erfüllen muss (Siehe \cite{RIntro} Abschnitt 6.3 \textit{Data-Frames} Absatz 1f): 
\begin{enumerate}
	\item Ein Data-Frame darf lediglich Vektoren, Matrizen, Faktoren und Data-Frames enthalten
	\item Alle Vektoren und Faktoren des Data-Frames müssen die selbe Länge besitzen, Matrizen zusätzlich eine einheitliche Breite
	\item Charakter- und String-Vektoren werden zu Faktoren vereinfacht
\end{enumerate} 
Data-Frames sind dahingehend wichtig, da eine Tabelle aus dem SQL-Server als Data-Frame interpretiert wird. 

\section{Machine Learning im SQL-Server 2017}
\label{sec:MLSQL} \label{sec:MachineLearning}
Innerhalb dieses Abschnittes befinden sich Code-Beispiele zur Umsetzung der in Kapitel \ref{cha:Theorie} vorgestellten Algorithmen. 

Es werden im Folgenden kurz die Einbindung der R-Skripte in TSQL behandelt, anschließend werden nur die R-Skripte für die einzelnen Punkte erläutert.

\paragraph{Verwendung von R im SQL-Server}
Um R im SQL-Server zu benutzen, wird die Stored Procedure \textit{sp\_execute\_external\_script} benötigt. Im Folgenden ein einfaches Beispiel: ~\newline

\begin{lstlisting}[language=SQL]
	EXECUTE sp_execute_external_script
	@language = N'R',
	@script = N' 
		mytextvariable <- c("hello", " ", input_data);
		OutputDataSet <- as.data.frame(mytextvariable);',
	@input_data = N' SELECT name FROM readers'
	WITH RESULT SETS (([Greetings] char(20) NOT NULL));
\end{lstlisting}

Hierbei wird in Zeile 2 zunächst die Sprache als Parameter übergeben, in Zeile 4 wird innerhalb des R Skriptes ein Begrüßungs-String erstellt, welcher in Zeile 5 als Ausgabe wiedergeben wird.

In Zeile 6 wird die Inputvariable definiert, an dieser Stelle sind SQL Befehle und gültige T-SQL Variablen möglich. Es können beliebig viele Inputvariablen definiert werden. 

In Zeile 7 wird die Ausgabe in Tabellenform überführt. Diese Zeile ist nicht zwingend notwendig.  

~\newline Dieses Schema bleibt allen Skript-Aufrufen gleich. Im Folgenden werden nur die R-Skripte vorgestellt.
\subsection{Lineare Regression}
Für diese Form der Regression gelten innerhalb des Paketes MicrosoftML folgende Bedingungen: ~\newline
\begin{enumerate}
	\item Strings und kalendarische Daten müssen über einen Faktor realisiert. werden 
	\item Der Ausgabewerte ist eine reelle Zahl.
\end{enumerate}

Um ein Modell für die lineare Regression zu erstellen, sind in R nur wenige Zeilen notwendig: ~\newline
\begin{lstlisting}[language=R]
	formel <- C ~ A+B;
	model <- rxLinMod(formula=formel, data=TrainingsData);
	serializedModel <- data.frame(payload = as.raw(serialize(model, connection=null)));
\end{lstlisting}

In der ersten Zeile wird zunächst eine allgemeine Formel definiert. Diese Formel ist zu interpretieren als $f: (A~x~B)\rightarrow C $ , das '+' ist hierbei nicht als Addition zu verstehen.

In Zeile 2 wird das Modell mithilfe der Bibliothek RevoscaleR und dem Methodenaufruf rxLinMod erstellt \textbf{und} trainiert. Als Parameter werden die Formel und die Trainingsdaten benötigt. 

In der dritten Zeile findet eine Serialisierung des Modells statt - dies ist nicht notwendig für eine direkte Verwendung, ermöglicht allerdings das Speichern des Modells innerhalb des SQL-Servers als Blob.  

Um das Modell anzuwenden, reichen ebenfalls wenige Zeilen R-Skript: \newline

\begin{lstlisting}[language=R]
	model <- unserialize(as.raw(serializedModel)); 
	C <- rxPredict(model,data.frame(TestData));
\end{lstlisting}

Hierbei wird zunächst in Zeile 1 das serialisierte Modell wieder nutzbar gemacht. 

In Zeile 2 wird die Methode \textit{rxPredict} der RevoScaleR-Bibliothek aufgerufen, welche aus den zu testenden Daten und dem Model eine Prognose erstellt. 
\subsection{Klassifikation}
Für die binäre Klassifikation mit RevoscaleR gelten folgende Bedingungen: ~\newline

\begin{enumerate}
	\item Die Klasse stellt einen Faktor mit Level 2 dar.
	\item Der Ausgabewert ist eine Wahrscheinlichkeit, mit der die Ausprägung positiv ausfällt
	\item Es kann gleichzeitig nur eine Klasse überprüft werden
\end{enumerate}

Der R-Code verhält sich parallel zum Code der linearen Regression:

\begin{lstlisting}[language=R]
	formel <- rain ~ temperature+humidity;
	logitmodel <- rxLogit(formula = form, data = TrainingsData);
	rainPropability <- rxPredict(model, data.frame(TestData));
\end{lstlisting}

Als Beispiel wurde hierbei die Voraussage gewählt ob es regnet, anhand von Temperatur und Luftfeuchtigkeit.

Eine Multiklassen-Klassifikation ist innerhalb der Standardbibliotheken von R nicht mithilfe logistischer Regression vorgesehen. Es gibt hierfür innerhalb der Umgebung allerdings die Möglichkeit, \textit{Random-Forests} zu verwenden.
\subsection{Neuronale Netze}
Es ist möglich, die im Abschnitt \ref{sec:NN} vorgestellten Konzepte direkt in R umzusetzen. Ein gutes Tutorial liefert hierbei \cite{SelbyNN}, welcher eine Schritt-Für-Schritt Anleitung und Erklärung bietet ein eigenes neuronales Netz zu entwerfen. Das Tutorial von Selby setzt einen ähnlichen Blogeintrag von Denny Britz (Siehe \cite{DennyNN}) in R um. 

Innerhalb dieser Arbeit wird allerdings das Paket \textit{MicrosoftML} verwendet.

\paragraph{Netz-Definition} ~\newline
Eine der wichtigsten Einstellung stellt die Definition des neuronalen Netzes dar. Für diese wird innerhalb der Microsoft-Umgebung (Innerhalb des ML-Servers, Azure und R-Services) einheitlich eine Definition in \textit{Net\#} verwendet. Diese Notation definiert  das gesamte neuronale Netz, und stellt einen einheitlichen und übertragbaren Standard in der Microsoft Umgebung dar. Ein einfaches Beispiel: ~\newline

\begin{lstlisting}[language=R]
 netDefinition <- ("
 	input Data auto;
 	hidden Hidden[25] sigmoid from Data all;
 	output Result[2] from Hidden all;  
 ")
\end{lstlisting}

In Zeile Zwei wird die Eingabeschicht mit dem Namen \textit{Data} und einer automatisch erkannten Größe erstellt. 

In Zeile Drei wird die versteckte Schicht \textit{Hidden} mit 25 Knoten, einer Verbindung zu allen Knoten in Data und der Aktivierungsfunktion \textit{Sigmoid} gewählt. 

In Zeile Vier wird die Ausgabeschicht \textit{Result} mit zwei Ausgabeknoten definiert. Es handelt sich um eine binäre Klassifikation. Die Aktivierungsfunktion wird auf den Standardwert \textit{sigmoid} gesetzt. 

Optional ist es möglich, die Größe eines hidden Layers in der Form [X,X,Y] anzugeben. Dies bedeutet, das zunächst zwei Layer mit X Knoten und anschließend ein Layer mit Y Knoten vorliegt, welche eine Einheit bilden. Die anderen Parameter, z.B. die Aktivierungsfunktion, werden für alle Teilschichten übernommen.

~\newline Es werden an ein neuronales Netzwerk innerhalb der .net\#-Deklaration folgende Anforderungen gestellt:

\begin{itemize}
	\item Jedes neuronale Netz besitzt mindestens eine Eingabeschichte und genau eine Ausgabeschicht.
	\item Die Anzahl der Knoten der Ausgabeschicht entspricht der Klasse des neuronalen Netzes (Ein Ausgabeknoten für Regression, zwei für binäre Klassifikation, \textit{n} für eine Klassifikation von \textit{n-}Labeln).
	\item Verbindungen müssen azyklisch sein, anders ausgedrückt, sie dürfen keine Kette von Verbindungen bilden, die zurück zum ursprünglichen Quellknoten führt.
	\item Um eine Vorhersage mit dem Modell zu machen, werden bei den Eingabedaten mindestens alle in der Formel angegebenen Features benötigt. 
\end{itemize}

~\newline Nach diesem Schema lassen sich beliebig komplexe neuronale Netze definieren. Es gibt weitere Möglichkeiten, die Netzdefinition anzupassen:

\begin{itemize}
	\item Auswahl von Aktivierungsfunktionen (z.B. Sigmoid, $tanh$, Linear)
	\item Deklaration von Konvolutionsbündeln, d.h. Schichten definieren, welche sich mit zusätzlichen Gewichten gegenseitig beeinflussen.
	\item Deklaration von Selektionsbündeln, d.h. Auswahlkriterien, nach welchen die Schichten verknüpft werden.
	\item Deklaration von Poolingbündeln, d.h. Schichten und Teilnetze, welche eine ähnliche Funktion erfüllen, allerdings nicht trainiert werden. 
\end{itemize} 
\newpage
\paragraph{Regression} ~\newline
Ein neuronales Netz mithilfe des Paketes zu erstellen, ist ähnlich einfach wie ein normales Modell hierfür:

\begin{lstlisting}[language=R]
netDefinition <- ("
	input Data auto;
	hidden Hidden[25] sigmoid from Data all;
	output Result[1] linear from Hidden all; ")
form <- C ~A+B;
model <- rxNeuralNet(
		formula = form, 
		data = TrainingsData,              
		type = "regression",
		netDefinition = netDefinition,
		numIterations = 100,
		normalize = "yes"
);
\end{lstlisting}

Zunächst wird ein Netz definiert, welches als Ausgabeschicht einen einzelnen Knoten mit linearer Ausgabefunktion besitzt. Anschließend wird die Formel aus dem obigen Beispiel für lineare Regression definiert, und das Modell mit der Funktion \textit{rxNeuralNet(...)} erstellt. Diese erhält gegenüber anderen Modellen zusätzliche Parameter für die Netzdefinition, die Iterationen und den Typ des neuronalen Netzes. 

Des Weiteren können Einstellungen über die Optimierungsfunktion, Initialisierung der Gewichte, sowie Lerngeschwindigkeit, Verfall und Beschleunigung vorgenommen werden.
\newpage
\paragraph{Multiclass-Labeling} ~\newline
Um eine Multiklassen-Klassifizerung vorzunehmen, benötigt man einen ähnlichen Aufbau:


\begin{lstlisting}[language=R]
netDefinition <- ("
		input Picture auto;
		hidden Hidden[250] sigmoid from Picture all;
		output Species[4] softmax from Hidden all; ")
form <- Species ~Sepal.Length+Sepal.Width+Petal.Length+Petal+Width;

model <- rxNeuralNet(
		formula = form, 
		data = TrainingsData,              
		type = "multiclass",
		netDefinition = netDefinition
);
\end{lstlisting}

Als Beispiel ist die Klassifizierung eines Bildes in eine von 4 Lotus-Spezien gewählt. Zu betonen ist, das die Ausgabe der Vorhersage 5 Werte erzeugt: Einen für die wahrscheinlichste Spezies, und für jede Spezies die Wahrscheinlichkeit. 

Die Implementierung eines binären Modells verhält sich analog, es muss innerhalb der Netzdefinition die Zeile 4 sowie der Typ des neuronalen Netzes in Zeile 10 entsprechend angepasst werden.
\section{Programmiersprache R}
\label{sec:R}
Die Sprache R stellt vorallem für Statistiker und Psychologen ein Standard-Tool dar, dennoch ist sie auch bei Data-Scientisten beliebt für ihre vielseitigen Plot- und Modellierungs-Möglichkeiten. Im Rahmen dieser Arbeit wurden alle Modelle mithilfe von R erstellt, was innerhalb dieses Abschnittes vorgestellt wird:

~\newline R ist eine Sprache sowie eine Entwicklungsumgebung für statistische Berechnungen und Grafiken. R ist ein GNU-Projekt und beruht auf der Sprache $S$, welche von John Chambers et al. entwickelt wurde. R stellt eine Implementation von S dar (vgl. \cite{RProject} Absatz 1) und liegt als Open Source Projekt vor.

~\newline R bietet eine große Bandbreite an statistischen Funktionen (z.B. Lineare und Nichtlineare Regression, Klassifikation und Signifikanztests) sowie grafische Aufbereitungen dieser und ist hochgradig Modular (vgl. \cite{RProject} Absatz 2). 

~\newline Die größten Stärken von R liegen neben der einfachen Anwendung statistischer Funktionen in der Aufbereitung als Plots. R erzeugt schnell verständliche Grafiken der Daten, bieten erfahrenen Nutzern allerdings viele Optionen exakt benötigte Darstellungen zu erzeugen. 

\paragraph{ R Umfeld} 
R umfasst folgende integrierte Dienste (Siehe \cite{RProject} Absatz 5f):
\begin{enumerate}
	\item Eine Speichereinheit und Daten-Engine
	\item Eine Umgebung für Berechnungen auf Listen, Vektoren und insbesondere Matrizen
	\item Eine Sammlung an Werkzeugen zur Datenanalyse, statistischen Auswertung und Erzeugung von Grafiken
	\item Eine Programmiersprache, welche auf Bedingungen, Schleifen und nutzerdefinierte Funktionen eingeht
\end{enumerate}

Für rechen- und Zeitintensive Operationen kann zusätzlich C und C++ Code zur Laufzeit eingebunden werden. Ebenso kann man mit C direkt Objekte manipulieren. 

~\newline Die Funktionalitäten von R können über ein Paket-System erweitert werden. Das wichtigste Paket innerhalb dieser Arbeit stellt \textit{RevoScaleR} dar, welches eine persistente Speicherung von Datenobjekten in Datenbanken und Files ermöglicht. 

\paragraph{Besonderheiten in der Programmierung}~\newline
Die wichtigste Besonderheit in R ist, das jedes Objekt als Vektor aufgefasst wird. Ein einzelner Wert wird ebenfalls als Vektor der Größe 1 betrachtet.

~\newline Vektoren können für arithmetische Ausdrücke verwendet werden, in diesem Fall werden die Operationen Element für Element ausgeführt. Zwei Vektoren, welche in einer Anweisung vorkommen, müssen nicht die selbe Länge besitzen. Falls dies nicht der Fall ist, ist die Ausgabe der Anweisung ein Vektor der Länge des längsten Vektors. Die kürzen Vektoren werden solange wiederholt, bis sie die Länge des längsten Vektors erreicht haben.

Insbesondere konstanten werden auf jedes Element angewendet (vgl. \cite{RIntro} Seite 13 Abschnitt 2.2 \textit{Vektorarithmetik} Absatz 1).

~\newline Diese Eigenschaft der Vektoren ist vor dem Hintergrund, mit Datenbanken zu arbeiten ein zweischneidiges Schwert: Zum einen werden die Operationen und Anweisungen sehr \textit{einfach} und Übersichtlich (Hilfsstrukturen für Schleifen entfallen), allerdings bringt v.A. die Wiederholung der kleineren Vektoren erhebliche Fehlerquellen mit sich. 

~\newline Ein Faktor ist ein Vektor mit einem fest definierten Wertebereich (z.B. ein Charakter-Vektor, Siehe auch \cite{RIntro} Kapitel 4 \textit{Ordered and Unordered Factors} Absatz 1).

~\newline Ein \textit{Array} stellt in R eine Kombination aus einem Wert-Vektor und einem Dimensions-Vektor dar. Der Dimensionsvektor gibt hierbei eine Form für den Wert-Vektor dar, und bestimmt in welcher Reihenfolge und ggfs. mit welchen Eigenschaften Operationen ausgeführt werden. Für arithmetische Operationen zweier Arrays wird ebenfalls die o.G. \textit{Recycling Rule} angewendet. Im Falle einer Anweisung eines Arrays und eines Vektors, wird zunächst versucht aus dem Vektor ein Array derselben Dimension zu erzeugen. Eine Matrix stellt ein zweidimensionales Array dar. 

~\newline Ein \textit{Data-Frame} stellt eine besondere Form einer Liste dar, die folgende Eigenschaften erfüllen muss (Siehe \cite{RIntro} Abschnitt 6.3 \textit{Data-Frames} Absatz 1f): 
\begin{enumerate}
	\item Ein Data-Frame darf lediglich Vektoren, Matrizen, Faktoren und Data-Frames enthalten
	\item Alle Vektoren und Faktoren des Data-Frames müssen die selbe Länge besitzen, Matrizen zusätzlich eine einheitliche Breite
	\item Charakter- und String-Vektoren werden zu Faktoren vereinfacht
\end{enumerate} 
Data-Frames sind dahingehend wichtig, da eine Tabelle aus dem SQL-Server als Data-Frame interpretiert wird. 

\section{Machine Learning im SQL-Server 2017}
\label{sec:MLSQL} \label{sec:MachineLearning}
Innerhalb dieses Abschnittes befinden sich Code-Beispiele zur Umsetzung der in Kapitel \ref{cha:Theorie} vorgestellten Algorithmen. 

Es werden im Folgenden kurz die Einbindung der R-Skripte in TSQL behandelt, anschließend werden nur die R-Skripte für die einzelnen Punkte erläutert.

\paragraph{Verwendung von R im SQL-Server}
Um R im SQL-Server zu benutzen, wird die Stored Procedure \textit{sp\_execute\_external\_script} benötigt. Im Folgenden ein einfaches Beispiel: ~\newline

\begin{lstlisting}[language=SQL]
	EXECUTE sp_execute_external_script
	@language = N'R',
	@script = N' 
		mytextvariable <- c("hello", " ", input_data);
		OutputDataSet <- as.data.frame(mytextvariable);',
	@input_data = N' SELECT name FROM readers'
	WITH RESULT SETS (([Greetings] char(20) NOT NULL));
\end{lstlisting}

Hierbei wird in Zeile 2 zunächst die Sprache als Parameter übergeben, in Zeile 4 wird innerhalb des R Skriptes ein Begrüßungs-String erstellt, welcher in Zeile 5 als Ausgabe wiedergeben wird.

In Zeile 6 wird die Inputvariable definiert, an dieser Stelle sind SQL Befehle und gültige T-SQL Variablen möglich. Es können beliebig viele Inputvariablen definiert werden. 

In Zeile 7 wird die Ausgabe in Tabellenform überführt. Diese Zeile ist nicht zwingend notwendig.  

~\newline Dieses Schema bleibt allen Skript-Aufrufen gleich. Im Folgenden werden nur die R-Skripte vorgestellt.
\subsection{Lineare Regression}
Für diese Form der Regression gelten innerhalb des Paketes MicrosoftML folgende Bedingungen: ~\newline
\begin{enumerate}
	\item Strings und kalendarische Daten müssen über einen Faktor realisiert. werden 
	\item Der Ausgabewerte ist eine reelle Zahl.
\end{enumerate}

Um ein Modell für die lineare Regression zu erstellen, sind in R nur wenige Zeilen notwendig: ~\newline
\begin{lstlisting}[language=R]
	formel <- C ~ A+B;
	model <- rxLinMod(formula=formel, data=TrainingsData);
	serializedModel <- data.frame(payload = as.raw(serialize(model, connection=null)));
\end{lstlisting}

In der ersten Zeile wird zunächst eine allgemeine Formel definiert. Diese Formel ist zu interpretieren als $f: (A~x~B)\rightarrow C $ , das '+' ist hierbei nicht als Addition zu verstehen.

In Zeile 2 wird das Modell mithilfe der Bibliothek RevoscaleR und dem Methodenaufruf rxLinMod erstellt \textbf{und} trainiert. Als Parameter werden die Formel und die Trainingsdaten benötigt. 

In der dritten Zeile findet eine Serialisierung des Modells statt - dies ist nicht notwendig für eine direkte Verwendung, ermöglicht allerdings das Speichern des Modells innerhalb des SQL-Servers als Blob.  

Um das Modell anzuwenden, reichen ebenfalls wenige Zeilen R-Skript: \newline

\begin{lstlisting}[language=R]
	model <- unserialize(as.raw(serializedModel)); 
	C <- rxPredict(model,data.frame(TestData));
\end{lstlisting}

Hierbei wird zunächst in Zeile 1 das serialisierte Modell wieder nutzbar gemacht. 

In Zeile 2 wird die Methode \textit{rxPredict} der RevoScaleR-Bibliothek aufgerufen, welche aus den zu testenden Daten und dem Model eine Prognose erstellt. 
\subsection{Klassifikation}
Für die binäre Klassifikation mit RevoscaleR gelten folgende Bedingungen: ~\newline

\begin{enumerate}
	\item Die Klasse stellt einen Faktor mit Level 2 dar.
	\item Der Ausgabewert ist eine Wahrscheinlichkeit, mit der die Ausprägung positiv ausfällt
	\item Es kann gleichzeitig nur eine Klasse überprüft werden
\end{enumerate}

Der R-Code verhält sich parallel zum Code der linearen Regression:

\begin{lstlisting}[language=R]
	formel <- rain ~ temperature+humidity;
	logitmodel <- rxLogit(formula = form, data = TrainingsData);
	rainPropability <- rxPredict(model, data.frame(TestData));
\end{lstlisting}

Als Beispiel wurde hierbei die Voraussage gewählt ob es regnet, anhand von Temperatur und Luftfeuchtigkeit.

Eine Multiklassen-Klassifikation ist innerhalb der Standardbibliotheken von R nicht mithilfe logistischer Regression vorgesehen. Es gibt hierfür innerhalb der Umgebung allerdings die Möglichkeit, \textit{Random-Forests} zu verwenden.
\subsection{Neuronale Netze}
Es ist möglich, die im Abschnitt \ref{sec:NN} vorgestellten Konzepte direkt in R umzusetzen. Ein gutes Tutorial liefert hierbei \cite{SelbyNN}, welcher eine Schritt-Für-Schritt Anleitung und Erklärung bietet ein eigenes neuronales Netz zu entwerfen. Das Tutorial von Selby setzt einen ähnlichen Blogeintrag von Denny Britz (Siehe \cite{DennyNN}) in R um. 

Innerhalb dieser Arbeit wird allerdings das Paket \textit{MicrosoftML} verwendet.

\paragraph{Netz-Definition} ~\newline
Eine der wichtigsten Einstellung stellt die Definition des neuronalen Netzes dar. Für diese wird innerhalb der Microsoft-Umgebung (Innerhalb des ML-Servers, Azure und R-Services) einheitlich eine Definition in \textit{Net\#} verwendet. Diese Notation definiert  das gesamte neuronale Netz, und stellt einen einheitlichen und übertragbaren Standard in der Microsoft Umgebung dar. Ein einfaches Beispiel: ~\newline

\begin{lstlisting}[language=R]
 netDefinition <- ("
 	input Data auto;
 	hidden Hidden[25] sigmoid from Data all;
 	output Result[2] from Hidden all;  
 ")
\end{lstlisting}

In Zeile Zwei wird die Eingabeschicht mit dem Namen \textit{Data} und einer automatisch erkannten Größe erstellt. 

In Zeile Drei wird die versteckte Schicht \textit{Hidden} mit 25 Knoten, einer Verbindung zu allen Knoten in Data und der Aktivierungsfunktion \textit{Sigmoid} gewählt. 

In Zeile Vier wird die Ausgabeschicht \textit{Result} mit zwei Ausgabeknoten definiert. Es handelt sich um eine binäre Klassifikation. Die Aktivierungsfunktion wird auf den Standardwert \textit{sigmoid} gesetzt. 

Optional ist es möglich, die Größe eines hidden Layers in der Form [X,X,Y] anzugeben. Dies bedeutet, das zunächst zwei Layer mit X Knoten und anschließend ein Layer mit Y Knoten vorliegt, welche eine Einheit bilden. Die anderen Parameter, z.B. die Aktivierungsfunktion, werden für alle Teilschichten übernommen.

~\newline Es werden an ein neuronales Netzwerk innerhalb der .net\#-Deklaration folgende Anforderungen gestellt:

\begin{itemize}
	\item Jedes neuronale Netz besitzt mindestens eine Eingabeschichte und genau eine Ausgabeschicht.
	\item Die Anzahl der Knoten der Ausgabeschicht entspricht der Klasse des neuronalen Netzes (Ein Ausgabeknoten für Regression, zwei für binäre Klassifikation, \textit{n} für eine Klassifikation von \textit{n-}Labeln).
	\item Verbindungen müssen azyklisch sein, anders ausgedrückt, sie dürfen keine Kette von Verbindungen bilden, die zurück zum ursprünglichen Quellknoten führt.
	\item Um eine Vorhersage mit dem Modell zu machen, werden bei den Eingabedaten mindestens alle in der Formel angegebenen Features benötigt. 
\end{itemize}

~\newline Nach diesem Schema lassen sich beliebig komplexe neuronale Netze definieren. Es gibt weitere Möglichkeiten, die Netzdefinition anzupassen:

\begin{itemize}
	\item Auswahl von Aktivierungsfunktionen (z.B. Sigmoid, $tanh$, Linear)
	\item Deklaration von Konvolutionsbündeln, d.h. Schichten definieren, welche sich mit zusätzlichen Gewichten gegenseitig beeinflussen.
	\item Deklaration von Selektionsbündeln, d.h. Auswahlkriterien, nach welchen die Schichten verknüpft werden.
	\item Deklaration von Poolingbündeln, d.h. Schichten und Teilnetze, welche eine ähnliche Funktion erfüllen, allerdings nicht trainiert werden. 
\end{itemize} 
\newpage
\paragraph{Regression} ~\newline
Ein neuronales Netz mithilfe des Paketes zu erstellen, ist ähnlich einfach wie ein normales Modell hierfür:

\begin{lstlisting}[language=R]
netDefinition <- ("
	input Data auto;
	hidden Hidden[25] sigmoid from Data all;
	output Result[1] linear from Hidden all; ")
form <- C ~A+B;
model <- rxNeuralNet(
		formula = form, 
		data = TrainingsData,              
		type = "regression",
		netDefinition = netDefinition,
		numIterations = 100,
		normalize = "yes"
);
\end{lstlisting}

Zunächst wird ein Netz definiert, welches als Ausgabeschicht einen einzelnen Knoten mit linearer Ausgabefunktion besitzt. Anschließend wird die Formel aus dem obigen Beispiel für lineare Regression definiert, und das Modell mit der Funktion \textit{rxNeuralNet(...)} erstellt. Diese erhält gegenüber anderen Modellen zusätzliche Parameter für die Netzdefinition, die Iterationen und den Typ des neuronalen Netzes. 

Des Weiteren können Einstellungen über die Optimierungsfunktion, Initialisierung der Gewichte, sowie Lerngeschwindigkeit, Verfall und Beschleunigung vorgenommen werden.
\newpage
\paragraph{Multiclass-Labeling} ~\newline
Um eine Multiklassen-Klassifizerung vorzunehmen, benötigt man einen ähnlichen Aufbau:


\begin{lstlisting}[language=R]
netDefinition <- ("
		input Picture auto;
		hidden Hidden[250] sigmoid from Picture all;
		output Species[4] softmax from Hidden all; ")
form <- Species ~Sepal.Length+Sepal.Width+Petal.Length+Petal+Width;

model <- rxNeuralNet(
		formula = form, 
		data = TrainingsData,              
		type = "multiclass",
		netDefinition = netDefinition
);
\end{lstlisting}

Als Beispiel ist die Klassifizierung eines Bildes in eine von 4 Lotus-Spezien gewählt. Zu betonen ist, das die Ausgabe der Vorhersage 5 Werte erzeugt: Einen für die wahrscheinlichste Spezies, und für jede Spezies die Wahrscheinlichkeit. 

Die Implementierung eines binären Modells verhält sich analog, es muss innerhalb der Netzdefinition die Zeile 4 sowie der Typ des neuronalen Netzes in Zeile 10 entsprechend angepasst werden.
\chapter{Fallbeispiel: Prognose von Taxifahrten}
\label{cha:Taxis} \label{cha:Experiment}
Innerhalb dieses Kapitels wird das Fallbeispiel der Taxidaten behandelt. Zunächst erfolgt eine Zielsetzung, anschließend in dem Abschnitt \ref{sec:Daten} eine Vorstellung des Versuchsaufbaus und der zugrunde liegenden Daten.

~\newline Hauptteil dieses Kapitels bildet in den Abschnitten \ref{sec:TipPred} bis \ref{sec:PasPred} die Ausarbeitung und Analyse verschiedener Prognosen mithilfe neuronaler Netze.

~\newline Abschließend findet sich in diesem Kapitel eine Übersicht über die herausgearbeiteten \textit{Best Practices} bei der Arbeit mit neuronalen Netzen im SQL-Server.

~\newline Auf die Aufnahme des Quellcodes wird aus Umfang verzichtet. Dieser findet sich in der angehängten CD.

\section{Ziele und Anforderungen}
\label{sec:Fallbeispiel} \label{sec:TaxiAllgemein}
\section{Eigenschaften der Daten}
\label{sec:Daten}
Innerhalb dieses Abschnittes werden zunächst die Daten vorgestellt, die dem Fallbeispiel zugrunde liegen. 

Die vorgestellten Daten haben bereits einen ETL-Prozess durchlaufen. Dieser besteht im Wesentlichen darin, die CSV-Dateien dahingehend aufzubereiten, das amerikanische Nummerierungen (z.B. Angabe von Dezimalzahlen mit '.' anstelle von ',') auf europäische Normen gebracht werden. Prinzipiell entfällt dieser Schritt für eine rein amerikanische Umgebung. 
\subsection{Taxifahrten}
\label{subsec:Taxidaten}
Zunächst werden die Daten der Taxifahrten erläutert. 

Diese stammen von der Stadt New York \cite{SourceTaxi} und wurde von der \textit{Taxi and Limousine Commission} (Kurz: TLC) bereitgestellt. 

Die TLC stellt einen Dachverband mehrerer Taxiunternehmen dar und veröffentlicht die Daten nur - die Erhebung erfolgt in einzelnen, anonymisierten Kleinunternehmen. 

~\newline Zusätzlich teilen sich die Fahrten in zwei Kategorien auf: \textit{Green} und \textit{Yellow}. Bei grünen Fahrten handelt es sich um Fahrzeuge mit einer anderen Lizenzierung (vgl. \cite{GreenTaxis} Absatz 5ff) und besonderen Auflagen. Im Allgemeinen verhalten sich die Fahrten allerdings gleich, insofern werden lediglich Unterschiede aufgelistet falls diese bestehen.
\paragraph{Attribute und Datentypen} ~\newline
Die folgende Übersicht entspricht der von der NYC bereitgestellten \cite{DataDicYellow}, die Beschreibung wurde übersetzt und eine Spalte für den Datentyp \footnote{Wie sie innerhalb des SQL-Servers bezeichnet werden} ergänzt. 
~\newline
\begin{center}
	\begin{tabular}{|p{0.3\textwidth}|p{0.5\textwidth}|p{0.1\textwidth}|}
		\hline
		Name & Beschreibung & Datentyp  \\ \hline
		\textbf{VendorID} & Ein Code für das Taxiunternehmen, welches die Daten bereitstellt & smallint \\ \hline
		pickup\_datetime & Uhrzeit und Datum, wann die Fahrt begann & datetime \\ \hline
		dropOff\_datetime & Uhrzeit und Datum, wann die Fahrt endete & datetime \\ \hline
		Passenger\_count & Anzahl der Fahrgäste & smallint \\ \hline	
		store\_and\_fwd\_flag & Angabe, ob die Fahrt direkt hochgeladen wurde, oder ob die Fahrt zwischengespeichert wurden vor einem Upload & bit \\ \hline
		RatecodeID & Ein Code für die Rate, welche für die Taxifahrt bezahlt wurde & smallint \\ \hline
		PULocationID & Ein Code für die Zone, in welcher die Fahrt begann & smallint \\ \hline
		DOLocationID & Ein Code für die Zone, in welcher die Fahrt endete & smallint \\ \hline
		trip\_distance & Distanzangabe des Taximeters & real \\ \hline
		fare\_amount & Der Fahrpreis berechnet aus Zeit und Distanz & \\ \hline
		extra & Verschiedene Zuschläge auf den Fahrpreis & real \\ \hline 
		MTA\_tax & Aufschlag, automatisch erhoben bei entsprechender Rate& real \\ \hline
		improvement\_surcharge & Aufschlag, automatisch erhoben in bestimmten Zonen & real \\ \hline
		payment\_type & Angabe des Zahlungsmittels als Code & smallint \\ \hline 
		tip\_amount & Höhe des Trinkgeldes & real \\ \hline
		tolls\_amount & Summierter Betrag von Zuschlägen dieser Fahrt& real \\ \hline
		total\_amount & Gesamtbetrag der Fahrt \textbf{ohne Trinkgeld} & real \\ \hline 
	\end{tabular}
\end{center}
~\newline
Die Daten der Grünen Taxis sind erweitert um einen Code für den \textit{Trip\_Type} (Ob eine Fahrt von einem Taxistand begann oder ob die Gäste an der Straße abgeholt wurden).

~\newline
Alle Distanz-Angaben entsprechen amerikanischen Meilen (1 mile$ \rightarrow $1,6 km), alle Währungsangaben Dollar. 

Für die Angaben der Codes sind ebenfalls Dictionaries bereitgestellt, diese spielen allerdings für den Machine-Learning-Aspekt dieser Arbeit keine Rolle und sind daher vernachlässigt worden. 
\paragraph{Umfang} ~\newline
Aus Ressourcengründen wurde ausschließlich das Jahr 2017 betrachtet. 

~\newline Es gibt \textbf{113 Millionen} Einträge für gelbe Fahrten, welche insgesamt knapp \textbf{8,1 GB Speicher} benötigen. Zusätzlich wurden Indizes angelegt mit weiteren 9,4 GB Speicher (Um schnelle Anfragen auf Uhrzeiten und Orte zu ermöglichen).

~\newline Es gibt \textbf{11,7 Millionen} Einträge für grüne Fahrten, mit insgesamt \textbf{900 MB Speicherplatz}. Es wurden zusätzlich Indizes mit 1,1 GB Speicher erstellt. 

~\newline Zusammen gibt es aus dem Jahr 2017 also fast \textbf{125 Millionen Einträge} welche insgesamt 19,5 GB Speicher belegen.
\paragraph{Anomalien} ~\newline
Innerhalb der Daten traten einige Ungewöhnlichkeiten auf - zum Beispiel gibt es Fahrten, die von Ort A nach Ort A gingen und keine Strecke zurückgelegt haben. Bei einer genaueren Untersuchung ergab sich allerdings, dass diese Fahrten meist wenige Minuten dauerten und ebenfalls keinen Passagier hatten. 

~\newline Es ist anzunehmen, das die Taxis an dieser Stelle auf ihre Passagiere gewartet haben. Aufgrund dieser Erkenntnis wurden alle Anomalien in die Machine-Learning Algorithmen übernommen, um die Daten und somit auch die Ergebnisse nicht zu verfälschen. 

Es wurden außerdem weitere Anomalien gefunden, welche kurz genannt werden:
\begin{itemize}
	\item Fahrten mit Negativkosten
	\item Fahrten außerhalb von 2017
	\item Fahrten mit extrem hohen Trinkgeld ($\sim 100$\$) oder extrem hohen kosten ($\sim300$\$)
	\item Fahrten, die wenige Sekunden gedauert haben
	\item Fahrten, welche mehrere Stunden gedauert haben und dabei nur kurze Strecken zurücklegen
\end{itemize}
\subsection{Wetteraufzeichnungen}
\label{subsec:Wetterdaten}
In diesem Unterabschnitt werden die Wetterdaten sowie ihr Umfang vorgestellt. 

~\newline Die Wetterdaten stammen von der \textit{National Oceanic and atmospheric Administration} \cite{SourceWeather} (Kurz: NOAA), welche verschiedene Klimadaten sammelt. Für dieses Fallbeispiel wurden die Wetterdaten der Wetterstation des JFK-Airports für das Jahr 2017 abgefragt. 




\newpage
\label{Prognosen}
\section{Trinkgeldprognose}
\label{sec:TipPred}
Als erstes Beispiel wird die Vorhersage des gegebenen Trinkgeldes behandelt. 

~\newline Dies stellt zwar nicht unbedingt ein für das Unternehmen hochgradig relevantes Thema dar, bietet jedoch einen guten Einstieg in das Thema Regression mit Hilfe von \textit{Deep Learning}:

~\newline Es ist anzunehmen, dass Passagiere nach einem gewissen Schema Trinkgeld geben, etwa um Beträge aufzurunden oder eine schnelle Fahrt zu \textit{belohnen}. Ebenso kann es Kriterien geben, welche nicht innerhalb der Daten erfasst sind, wie die Stimmung der Passagiere oder die Freundlichkeit des Fahrers. Genauso treten eventuell Kriterien auf, welche zu einem Entfall des Trinkgeldes führen, eine lange Fahrt, eine Panne, oder ein Taxifahrer der \textit{Extrarunden dreht}. 

~\newline Dennoch lassen sich diese potenziellen Kriterien schwer in einen anwendbaren Katalog zusammenfassen, nachdem man logisch das Trinkgeld erschließen kann. Vor diesem Hintergrund bietet sich die Verwendung eines tiefen Neuronalen Netzes an: 

Die erste versteckte Schicht extrahiert (uns unbekannte, nicht genauer definierte) Kriterien automatisch, und die zweite Schicht gewichtet diese. 
\paragraph{Zielsetzung} ~\newline
Das Modell soll vorhersagbare Attribute erhalten, um für eine geplante Fahrt das Trinkgeld zu schätzen. 

Ziel ist es eine gute Prognose mit realistischen, d.h. abschätzbaren Features zu erzeugen. 

~\newline Aus unternehmerisches Sicht ist dieser Use-Case relevant, da Fahrer voraussichtlich die Fahrten mit höherem Trinkgeld bevorzugen werden, obwohl sich diese nicht mit dem größten Nutzen des Unternehmens decken müssen. 
\paragraph{Versuchsaufbau} ~\newline
Die Fahrten liegen in gefilterter, aber nicht weiter aufbereiteter Form vor. 

Als Eingabefeatures werden die Fahrtdauer in Minuten, der Start- und Ziel-Ort, die Kosten der Fahrt, die gefahrene Strecke und die Anzahl der Passagiere gewählt. 

Die Ausgabe ist eine Schätzung des Trinkgeldes in Dollar.
\paragraph{Netzaufbau und Einstellungen} ~\newline
Das neuronale Netz wurde mit zwei Schichten á 100 Knoten definiert. Die Schichten sind vollvermascht und verwenden beide die Sigmoidfunktion. Als Ausgabefunktion wurde \textit{Linear} gewählt.

~\newline Die Lernrate wurde auf 5\% gesetzt und als Optimierungsfunktion Stochastic Gradient Descent gewählt. Es werden jeweils 500 Epochen durchgeführt. 
\paragraph{Tuning und Verbesserung} ~\newline
Wesentliche Unterschiede in der Genauigkeit dieses Beispiels verbucht die Auswahl der Features:
\begin{itemize}
	\item Modelle, welche nicht die Gesamtkosten der Fahrt als Eingabe erhalten, erzielen nur Genauigkeiten von 75\%.
	\item Ein Modell, welches sich auf ausschließlich die Kosten der Fahrt stützt, erzielte eine Genauigkeit von ca. 40\%. 
	\item Werden die im Versuchsaufbau genannten Features um (grobe) Wetterdaten und eine Tageszeit erweitert, verbessert sich die Genauigkeit des Modells auf bis zu 98\%. 
	\item Werden die im Versuchsaufbau genannten Features um die Passagieranzahl gekürzt wird nur eine Genauigkeit von ca. 80\% erzielt - die Passagieranzahl scheint also ein wesentliches Feature darzustellen.
	\item Weitere Features, welche sich auf den Preis beziehen (z.B. die Rate der Fahrt, Steuerliche Zulagen, diverse Aufschläge), welche bereits im Gesamtpreis enthalten sind, bringen keine Veränderung der Genauigkeit. 
\end{itemize} 
~\newline Im Rahmen der Zielsetzung wurden aber lediglich die unter dem Versuchsaufbau gewählten Attribute verwendet. Die Vorhersage des Wetters gestaltet sich weitestgehend schwierig für ein Taxiunternehmen bzw. den Fahrer. 

~\newline Weitere Unterschiede erzeugten Veränderungen in der Netzdefinition: 

Eine Verkleinerung einer der beiden Schichten unter 50 Knoten führte zu drastisch schlechteren Ergebnissen (max. 60\% Genauigkeit). 

Eine Vergrößerung der Schichten über 100 Knoten führte zu keinem Gewinn an Genauigkeit, jedoch zu längeren Laufzeiten (Ein Trainingsdurchlauf mit je 200 Knoten dauert ca. 4 mal so lange).

~\newline Eine Verwendung der Aktivierungsfunktion \textit{tanh} führte zu keinen Unterschieden, eine Verwendung der Softmax-Aktivierungsfunktion führte zu einer um ca. 20\% längeren Laufzeit und ähnlichen Genauigkeiten. 
\paragraph{Ergebnisse} ~\newline
Die folgenden Ergebnisse wurden mithilfe der unter \textbf{Netzaufbau und Einstellungen} genannten Optionen erzielt:

\begin{figure}[h]
	\begin{center}
		\includegraphics[width=0.95\linewidth]{Bilder/TrinkgeldErgebnisse}
		\caption[Ergebnisse der Trinkgeldprognose]{Ergebnisse der Trinkgeldprognose}
		\label{fig:TipErg}
	\end{center}
\end{figure}

\todo{"Minifazit" hier? Kurze bewertung oder einfach so stehen lassen?}
\newpage
\section{Ratenerkennung}
\label{sec:RatePred}
Als zweites Beispiel wird die Erkennung der Fahrrate behandelt. Dieser Fall ist weitestgehend Trivial, wurde dennoch als einfache Referenz für Multiklassen-Klassifikation aufgenommen. 

~\newline Im Gegensatz zur Trinkgeldprognose sind die Kriterien, nach welchen sich die Fahrtkosten berechnen, klar dokumentiert: Je nach Rate kostet jede Meile und jede Minute einen bestimmten Betrag. Zusätzlich kommt je nach Gebiet (z.B. JFK-Airport) ein Aufschlag hinzu. Dahingehend lässt sich diese Formel zur Erkennung der Rate umformen.

~\newline Interessant wird dieser Use-Case vor dem Hintergrund, das die Raten bzw. ihre zugehörigen Verrechnungssätze nicht innerhalb der Daten auftreten, und das Modell erkennen muss, welche Beträge Aufschläge sind, und welche Beträge die Raten ausmachen. 

Das Modell soll also in diesem Fall einen für den Menschen eindeutigen, klar nachvollziehbaren Sachverhalt reproduzieren. 
\paragraph{Versuchsaufbau} ~\newline
Die Fahrten liegen in gefilterter, aber nicht weiter aufbereiteter Form vor. 

Als Eingabefeatures werden die Fahrtdauer in Minuten, der Start- und Ziel-Ort, die Kosten der Fahrt, die gefahrene Strecke sowie die Anzahl der Passagiere gewählt. 

Die Ausgabe ist die Rate der Taxifahrt als Dictionary-ID.
\paragraph{Netzdefinition} ~\newline
Das neuronale Netz wurde mit zwei Schichten á 100 Knoten definiert. Die Schichten sind vollvermascht und verwenden beide die Sigmoidfunktion. Als Ausgabefunktion wurde \textit{Softmax} gewählt.

~\newline Die Lernrate wurde auf 5\% gesetzt und als Optimierungsfunktion Stochastic Gradient Descent gewählt. Es werden jeweils 250 Epochen durchgeführt. 
\paragraph{Tuning und Verbesserung} ~\newline
Innerhalb der Ratenerkennung ließen sich wenig Verbesserungen vornehmen: Grund hierfür ist die bereits anfänglich hohe Genauigkeit. 

Eine Verbesserung der Geschwindigkeit lässt sich mit einem kleineren Netz erzielen. Eine Netzdefinition mit 60 und 10 Knoten in den Hidden Layern erzielte bei größeren Trainingsmengen ebenfalls eine Genauigkeit von 99\%, was für die meisten Anwendungen ausreichend ist. Es benötigt allerdings nur ca. ein Fünftel der Zeit fürs Training.
\paragraph{Ergebnisse} ~\newline
Die unter Netzdefinition genannten Optionen erzielten folgende Werte:

\begin{figure}[h]
	\begin{center}
		\includegraphics[width=0.95\linewidth]{Bilder/RatenErgebnisse}
		\caption[Ergebnisse der Trinkgeldprognose]{Ergebnisse der Ratenerkennung}
		\label{fig:RateErg}
	\end{center}
\end{figure}

~\newline Ein wichtiges Teilergebnis ist die Veränderung des ersten Versuches mit 1000 Trainingsdaten zu dem mit 5000 Trainingsdaten: 
~\newline Zwar steigt die Genauigkeit nur minimal, dennoch werden erst ab 5000 Trainingsdaten die weniger vertretenen Daten korrekt erkannt. Das Modell für 1000 Trainingsdaten gab immer die Rate \textit{Standard} aus - welche zu 97\% auftritt. 


~\newline Von den (zuletzt) nicht erkannten Daten besitzen 75\% die Rate \textit{Negotiated}.

~\newline Als Nebenergebnis ist zu nennen, das die Trainingsverfahren meist nach ca. 100 Epochen aufhören, weil keine weiteren Änderungen in der Genauigkeit erzielt werden.
\newpage
\section{Fahrtenaufkommen}
\label{sec:RidesPred}
Als nächstes Beispiel wird die Prognose des Fahrtenaufkommens behandelt. 

~\newline Diese Prognose stellt die Erste für aggregierte Daten dar, und besitzt somit wesentlich weniger Trainingsdaten.

~\newline Eine weitere Schwierigkeit für das Modell besteht darin, das die Testdaten aus der Menge der Trainingsdaten entfernt wurden, und somit die Gruppierungen der Testdaten in dieser Form nicht vorliegen. 

Das Modell muss also, um die Fahrten des 31. August um 12:00 vom JFK-Airport vorherzusagen,welche in dieser Kombination nicht im Training vorkommen, einen Transfer auf unbekannte Daten leisten. Das Neuronale Netz kann sich in diesem Fall nicht \textit{erinnern}.

~\newline Eine letzte Schwierigkeit für dieses Modell stellt die große Reichweite der Werte dar: Während es in den Äußeren Zonen häufig zu wenigen oder keinen Fahrten kommt, werden Börsenviertel und Flughäfen stark frequentiert. 

Hierbei gibt es sowohl in den Orten als auch in den kalendarischen Daten starke Ausschläge, welche innerhalb der gruppierten Daten zu einer breiten Streuung führt.
\paragraph{Zielsetzung} ~\newline
Ziel des Modells ist es, für einen gegebenen Ort und eine gegebene Stunde eines Tages die Anzahl der \textbf{ausgehenden} Fahrten vorherzusagen. 

~\newline Dieses Modell hilft dem Unternehmen, seine Taxi-Kontingente strategisch zu verteilen bzw. kann dem Taxifahrer helfen, vor einer Rushhour am richtigen Taxistand einzutreffen. 
\paragraph{Versuchsaufbau} ~\newline
Die Taxifahrten liegen in aggregierter Form vor: Sie wurden nach Datum, Tageszeit auf Stunde gerundet und Startort gruppiert. Die daraus Resultierenden Trainingsdaten sind Wertepaare aus der Anzahl der Fahrten, Ort, Stunde und Datum. 

Die Gruppierungskriterien werden in R als Faktoren repräsentiert. 

Ausgabe des Modells ist eine Schätzung der Fahrten, welche auf eine Ganzzahl gerundet wird. 

\paragraph{Netzdefinition} ~\newline
Für die endgültige Netzdefinition wurde ein einzelner Layer mit 300 Knoten und der \textit{tanh}-Aktivierungsfunktion gewählt. Die Ausgabe erfolgt über die \textit{Linear}-Funktion.

~\newline Die Lernrate wurde auf 15\% gesetzt und als Optimierungsfunktion Stochastic Gradient Descent gewählt. Es wurde ein Verfall von 5\% gewählt. Es werden jeweils 500 Epochen durchgeführt. 
\paragraph{Tuning und Verbesserung} ~\newline
Die Prognose des Fahrtenaufkommen hat in den ersten Versuchen mit zwei Hidden Layern keine Ergebnisse erzielt (genau genommen negative $r^2$-Werte \footnote{negative $r^2$-Werte bedeuten, das eine Gerade bessere Werte erzielt, als das Modell}). Das erzeugte Modell gab in jedem Fall den Mittelwert der Fahrten aus, die Gewichte innerhalb des Modells hatten verschwindend geringe Werte und der Mittelwert wurde als Bias addiert. 


~\newline Hierfür gibt es zwei mögliche Erklärungen: Die erste ist Overfitting. Die zweite ist eine mögliche, noch zu geringe Komplexität bzw. fehlende Features. 

Um das Problem zu analysieren wurden dahingehend verschiedene Netzparameter durchgespielt, von [10,10] bis [1000,1000]-Netzen wurde eine große Bandbreite getestet. Auch die Veränderung der Aktivierungsfunktionen zu Sigmoid, Tanh, Softmax und Linear, bzw. Kombinationen dieser, erzeugte keine Ergebnisse. 

~\newline Die ersten Ergebnisse wurden erzielt, als die zweite versteckte Schicht entfernt wurde. Das neue Modell mit einer einzelnen, 100 Knoten großen Schicht zeigte das zu erwartende, trainierbare verhalten. Hier zeigten größere Trainingsdaten sowie mehr Epochen eine Positive Wirkung auf die erstmals positive Genauigkeit. Ebenso zeigten unterschiedliche Aktivierungsfunktionen verschiedene Ergebnisse (Linear sehr schlechte, Softmax und Sigmoid gute, Tanh sehr gute). 

~\newline Dieses Verhalten legt nahe, das es sich um eine Art von Overfitting gehandelt hat. 

~\newline Die besseren Ergebnisse der Tanh-Aktivierungsfunktion gegenüber der Sigmoid- und Softmax-Funktion sind voraussichtlich auf das Verhalten um die Nullstelle zurückzuführen: Während die $sig(0)=0.5$ ist, gilt $tanh(0)=0$. Somit werden Null-Werte, wie sie in den Faktorisierten Werten (z.B. Ort) häufig auftreten, besser abgebildet und werden unabhängig vom angelegten Gewicht nicht in die weiteren Berechnungen aufgenommen. 

Unter Verwendung der Sigmoid-Funktion werden bei vielen Epochen und Trainingsdaten zwar ebenfalls akzeptable Ergebnisse erzielt, dennoch liegt das i.A. daran, dass die Gewichte der Faktorisierten Daten gegen Null reduziert werden. Dies führt dazu, das faktorisierte Features nicht (stark) in die Prognose aufgenommen werden, und durch den \textit{Wegfall} der Features ein schwächeres Modell erzeugt wird. 

~\newline Eine Verbesserung der Genauigkeit hat sich ebenfalls durch das Erhöhen der Neuronenanzahl sowie der Anzahl der Epochen ergeben. Die in Netzdefinition genannten Einstellungen ergaben hierbei die kleinsten Werte mit der höchsten erzielten Genauigkeit.

~\newline Eine potenzielle Verbesserung würde eine komplexere Netzdefinition darstellen, welche zwei (oder drei) parallele Schichten besitzt: Eine für die numerischen Daten mit der Sigmoid- oder Softmax-Funktion sowie eine getrennte Schicht für die faktorisierten Daten. 
~\newline Was (meiner Meinung) nach ebenfalls eine wesentliche Verbesserung darstellen kann ist die Aufnahme eines zweiten Jahres in die Trainingsdaten.
\paragraph{Ergebnisse} ~\newline
Die unter Netzdefinition genannten Optionen erzielten folgende Werte:

\begin{figure}[h]
	\begin{center}
		\includegraphics[width=0.95\linewidth]{Bilder/FahrtenErgebnisse}
		\caption[Ergebnisse der Trinkgeldprognose]{Ergebnisse der Fahrten-Prognose}
		\label{fig:RidesErg}
	\end{center}
\end{figure}

Dies stellt zwar nicht die besten Ergebnisse dar, dennoch stellen die unter \textbf{Tuning} vorgestellten Maßnahmen die wichtigsten praktischen Erkenntnisse dieser Arbeit dar. 

~\newline Bei Eintausend Trainingsdaten gibt das Modell lediglich den Mittelwert aus - es liegen noch zu wenig Trainingsdaten vor. Bei Fünf- und Zehntausend liegt zunächst ein starker Einbruch vor, was darauf zurückzuführen ist, dass sich das Modell an den (wenigen) Trainingsdaten ausrichtet, die inhaltlich zu weit entfernt sind von den tatsächlichen Testdaten (Es könnten z.B. keine Daten für den Oktober, oder keine Daten für Nachtfahrten ins Training eingeflossen sein).

~\newline Werden die Ergebnisse nicht gerundet (auf \textit{ganze Fahrten}) so liegen die $r^2$-Werte ca. 2\% höher (in allen Versuchen).
\newpage
\section{Passagieranzahl}
\label{sec:PasPred}
Ein weiteres Beispiel ist die Vorhersage der Anzahl der Passagiere, welche an einer Fahrt teilnehmen. 

Dieser Use-Case ist eher für technische Belange interessant, denn die Verteilung der Passagieranzahl\footnote{Es handelt sich hierbei um die ungefilterten Daten} über die Fahrten stellt eine Besonderheit dar: 


\begin{figure}[h]
	\begin{center}
		\includegraphics[width=0.8\linewidth]{Bilder/PassagierVerteilung}
		\caption[Verteilung der Passagieranzahl]{Verteilung der Passagieranzahl über alle Fahrten}
		\label{fig:PassagierVerteilung}
	\end{center}
\end{figure}


Wie zu erkennen ist, besitzen die meisten Fahrten einen einzelnen Passagier. Diese machen Rund 70\% der Fahrten aus.

Von den verbleibenden Fahrten machen Zwei-Passagier-Fahrten ca. 60\% aus.  Dies hat sich im Verlauf der Entwicklung als überaus herausfordernd herausgestellt. 
\paragraph{Zielsetzung} ~\newline
Ziel des Modells ist es, für einen gegebenen Ort, Datum und Stunde einer Fahrt die Anzahl der Passagiere vorherzusagen. 
 
\paragraph{Versuchsaufbau} ~\newline
Für diesen Use-Case wurden verschiedene Versuche durchgeführt. Gemeinsamkeit bilden die Eingegebenen Daten: Datum, Tageszeit und Ort des Fahrtbeginns. 
\begin{enumerate}
	\item In \textit{Variante 1} wurde das Modell mit Regression erstellt. 
	\item In \textit{Variante 2} wurde das Modell für Multiklassen-Klassifizierung erstellt.
	\item In \textit{Variante 3} wurden Einzelfahrten gefiltert und ein Regressionsmodell erstellt. 
	\item In \textit{Variante 4} wurden ebenfalls Einzelfahrten gefiltert und ein Multiklassen-Modell generiert. 
\end{enumerate}
Ausgabe war, abhängig von der Variante, die Anzahl der Fahrten als Ganzzahl (1 \& 3) oder Klasse (2 \& 4). 

Es wurden verschiedene Netzdefinitionen durchgespielt (Was Tiefe, Schichtgröße und Aktivierungsfunktionen angeht), dennoch erzielten diese für alle 4 Varianten die selben Ergebnisse:
\paragraph{Ergebnisse} ~\newline
Variante 1 \& 2 erreichten, bei größeren Trainingsmengen (ca. 10000), eine Genauigkeit von 70\% \footnote{Variante 1 nur bei einer Rundung auf Ganzzahl, ansonsten etwa 65\%}. Während dies Objektiv ein akzeptabler Wert ist, gibt das Modell für jede Eingabe stets eine \textit{Ein-Personen-Fahrt} aus. Diese Annahme stimmt in 70\% aller Fälle. 

~\newline Variante 3 \& 4 zeigten dieselbe Schwäche für \textit{Zwei-Personen-Fahrten}. Für kleinere Trainingsdaten wurden Modelle erzeugt, welche andere Ausgaben erzeugten, allerdings wurden mit zunehmenden Iterationen und Trainingsdaten wieder nur die statistisch häufigsten Werte ausgegeben. 

~\newline Die dabei entstehenden Modelle zeichnen sich alle durch verschwindend geringe Gewichte aus. Die einzige Änderung geschieht innerhalb des Bias, welcher in beiden Modellen (nach Abschluss des Trainings) dem Ergebnis entspricht.

~\newline Da auch die Verwendung von der $Tanh$-Funktion und der Linear-Funktion in keinem befriedigendem Maße Ergebnisse zeigte (Welche sich für stark von Faktoren abhängige Schichten besser zu eignen scheinen, Siehe \ref{sec:RidesPred}), ist anzunehmen, dass elementare Features fehlen. Ein oder mehrere ausschlaggebende Kriterien, um die Passagieranzahl zu ermitteln, wurde innerhalb der Trainingsdaten noch nicht abgebildet/berücksichtigt. 

~\newline Ein Nebenergebnis des Versuches ist der direkte Vergleich der Geschwindigkeiten: Variante 1 und 3, sowie Variante 2 und 4 verhielten sich in ihrer Trainingsdauer jeweils Analog. 

Die Multiklassen-Varianten benötigten aber ca. die 3 fache Zeit ihrer Regressions-Gegenstücke.

\begin{figure}[h]
	\begin{center}
		\includegraphics[width=0.8\linewidth]{Bilder/PassagierErgebnisse}
		\caption[Ergebnisse der Passagiervorhersage]{Ergebnisse der Passagiervorhersage (Variante 2)}
		\label{fig:PasErg}
	\end{center}
\end{figure}
 
 
 Die oben stehenden Ergebnisse in Abbildung \ref{fig:PasErg} entsprechen den Zeiten von 250 Iterationen bei einem einzelnen $tanh$-Layer mit 250 Knoten und einer Multiklassen-Ausgabe. 
\newpage
\section{Umsatzvorhersage}
\label{sec:RevPred}
Als letztes Beispiel wird die Umsatzvorhersage behandelt. Dieses stellt für das Unternehmen offensichtlich den wichtigsten Use-Case dar.Sie gibt dem Unternehmen nicht nur eine Übersicht über den ist-Zustand, sondern zeigt ebenfalls detailliert an welchen Punkten noch gearbeitet werden muss. 

Ein gut funktionierendes Modell kann ggfs. auch zur Prognose von Änderungen erweitert werden.
\paragraph{Zielsetzung} ~\newline
Ziel des Modells ist es, für einen gegebenen Ort und eine gegebene Stunde eines Tages Summe der Fahrtenkosten auszugeben. 

\paragraph{Versuchsaufbau} ~\newline
Der Gesamtumsatz wird aus den Einzelfahrten aufsummiert: Sie wurden nach Datum, Tageszeit auf Stunde gerundet und Startort gruppiert. Die daraus Resultierenden Trainingsdaten sind Wertepaare aus dem Gesamtumsatz, Ort, Stunde und Datum. 

Die Gruppierungskriterien werden in R als Faktoren repräsentiert. 

Ausgabe des Modells ist eine Schätzung des Umsatzes, welche auf zwei Nachkommastellen gerundet wird. 

\paragraph{Netzdefinition} ~\newline

\paragraph{Versuchsaufbau} ~\newline

\paragraph{Netzdefinition} ~\newline

\paragraph{Tuning und Verbesserung} ~\newline

\paragraph{Ergebnisse} ~\newline

\section[Best Practices]{Best Practices bei neuronalen Netzen im SQL-Server}

Im Rahmen des Praxisteils haben sich im Wesentlichen 3 Best Practices herauskristallisiert, welche nun genauer vorgestellt werden. 

\subsection{Auschnitts-Tabellen und Aggregatstabellen} ~\newline
Im Rahmen des Trainings mussten zufällige Daten ausgewählt werden. Dies wird innerhalb von SQL über die Sortierung einer zufälligen \textit{ID} realisiert. 

~\newline Dieses Verfahren dauert für die vorliegenden Daten (113 Millionen Datensätze) ca. 26 Minuten \footnote{Bei Verwendung eines einfachen Desktop-PCs}. Während diese Zahl bei Modellen, die Stunden trainieren, eher unbedeutend ist, stellte sie gerade für die ersten \textit{Testmodelle} einen Großteil der benötigten Zeit dar. 

Eine wesentliche Verbesserung ergab sich durch die Erzeugung einer (kleineren) Tabelle, die bereits zufällig sortiert ist, und ein Training mithilfe dieser Tabelle. 

~\newline Vor allem für kleinere Modelle, welche lediglich mit wenigen Tausend Daten trainiert wurden, führte dies zu einer markanten Beschleunigung.

~\newline Bei den Modellen, welche sich auf aggregierte Daten stützen, fiel dieser Effekt noch größer aus: Eine Aufsummierung des Umsatzes nach Ort und Stunde dauerte ca. eine Stunde. Hierfür wurde ebenfalls eine (redundante) Tabelle erstellt, welche bereits (mehrere) aggregierte Werte hält. 

\subsection{ML-Templates} ~\newline
Nach den ersten Schritten mit den neuronalen Netzen zeigte sich ein Muster, welches sich auf alle Use-Cases übertragen lies. 

Dieses Muster lässt sich mithilfe von vier SQL-Dateien darstellen:

\begin{enumerate}
	\item Eine Datei zur Erstellung einer Prozedur, welche das neuronale Netz erstellt, trainiert und abspeichert.
	
	Hier findet sich die Netzdefinition, Trainingsparameter und ggfs. eine Aufbereitung der Trainingsdaten.
	\item Eine Datei zur Erstellung einer Prozedur, welche alle Testdaten mithilfe des neuronalen Netzes vorhersagt.
	
	Hier findet sich eine analoge Aufbereitung der Testdaten zu den Trainingsdaten.
	\item Eine Datei zur Erstellung einer Prozedur, welche die Vorhersage bewertet. Je nach Art des Netzes wird die Genauigkeit berechnet und eine Stichprobe der Vorhersagen genommen.
	
	Diese Prozedur lässt sich soweit vereinfachen, das es jedes beliebige Modell bewertet.
	\item Eine Datei, welches die Prozeduren ausführt.
\end{enumerate}

Mithilfe dieser Templates ließen sich sehr schnell Use-Cases umsetzen und die Modelle sowie Prozeduren waren einheitlich und übersichtlich. 

Für eine reelle Anwendung sollten noch zwei weitere Prozeduren aufgenommen werden: Eine zum \textit{Weitertrainieren} eines Modells, sowie eine zur konkreten Vorhersage eines einzelnen Datensatzes. 

\subsection{Speicherung der Modelle} ~\newline
Für die Organisation, den Vergleich der Modelle und die allgemeine Verwendung wurden die serialisierten Modelle in einer eigenen Tabelle abgespeichert. 

Während diese zunächst nur das Binärfile und den Namen hielten, war dies nicht wirklich hilfreich bei der Wiederverwendung. Deswegen wurde die Tabelle um folgende Eigenschaften erweitert, welche innerhalb der oben genannten Prozeduren mitgesetzt werden:

\begin{enumerate}
	\item Das Erstellungsdatum
	\item Die Art des Modells (Klassifizerung oder Regression)
	\item Die Anzahl der Trainingsdaten
	\item Die Genauigkeit
	\item Einen Kommentar
\end{enumerate}

Die ersten 3 werden hierbei von der Trainings-Prozedur erstellt, die Genauigkeit durch die Bewertungs-Prozedur.

Diese erweiterte Tabelle stellte sich als wesentlich benutzerfreundlicher heraus und lässt sich über das Template problemlos warten. 
\chapter{Fazit}
\label{cha:Fazit} \label{cha:Schluss}
Im Rahmen der Arbeit haben sich verschiedene, übergreifende Ergebnisse herauskristallisiert welche nun abschließend aufbereitet werden.

~\newline In Bezug auf die Benutzung der Algorithmen gilt: \textit{\textbf{Easy to learn, hard to master}}. 

~\newline Moderne Frameworks und Bibliotheken ermöglichen es \textit{kinderleicht} komplexe Modelle zu erstellen und zu benutzen. Sogar die Aufbereitung der Daten erfolgt einfach mithilfe optischer Werkzeuge oder wird sogar automatisch übernommen (z.B. die Normierung und Faktorisierung innerhalb von RevoScaler).

~\newline Dennoch liefern diese Modelle nicht unbedingt zufriedenstellende Ergebnisse: 

Entweder sind die Standard-Einstellungen nicht geeignet, oder die Aufbereitung der Daten produziert neue Fehler (z.B. wenn die Orte als ID übergeben werden, sind diese nicht faktorisiert). Auch entstehen häufig Fehler bei automatisch faktorisierten Daten, etwa weil innerhalb der Trainingsmenge eien Merkmalsausprägung gefehlt hat, welche in den Testdaten vorkommt und nicht korrekt auf den Eingabevektor übertragen werden kann. 

Fehler dieser Art zeigen sich erst bei der konkreten Anwendung des Modells auf separate Testdaten: Werden die \textit{korrumpierten} Trainingsdaten und Einstellungen in Standardeinstellungen verwendet, so zeigt das Modell bei zunehmenden Training ebenfalls Erfolg - es spiegelt nur nicht zwangsläufig echte Erfolge wieder.

~\newline Insofern sind die Frameworks und Standardeinstellungen zwar angenehm für den Einstieg, sollten allerdings für die Umsetzung eines Projektes genau inspiziert werden. Vor allem die Aufbereitung der Trainingsdaten und die Kontrolle des Modells sollte der Entwickler selbst übernehmen.  

~\newline Ein häufiger Satz, welcher v.A. in Internetforen auftritt, ist \textit{\textbf{building neural networks is more art than science}}.

Diesen Satz sollte man nicht unreflektiert hinnehmen - viele Änderungen, Designentscheidungen und Trainingserfolge wurden bewusst durch die wissenschaftlichen Eigenschaften der einzelnen Komponenten erzielt. 

~\newline  Konkret zeigte sich dies bei der Benutzung der $tanh$-Funktion gegenüber der Sigmoid-Funktion in Abschnitt \ref{sec:RidesPred} welche Aufgrund ihrer Eigenschaften mit negativen Werten bessere Ergebnisse erzielte. 

~\newline Auch ist die Analyse des Datenbestandes, im Speziellen die Suche nach Anomalien, und das Design von Features im Bereich der Wissenschaft anzusiedeln. 

~\newline Zwar sind viele Fortschritte, z.B. das beheben von Over- und Underfitting, experimentell entstanden, dennoch können 
grundlegende Probleme auftreten, welche sich nicht (zeitnah) experimentell lösen lassen, im Rahmen dieser Arbeit konkret Abschnitt \ref{sec:RevPred} zur Schätzung des Umsatzes und \ref{sec:PasPred} für das Passagieraufkommen. Hier haben selbst umfangreiche Experimente keine Ergebnisse erzielt.

~\newline  Als letzter, großer Themenblock \textit{pro science} ist ebenfalls das Performancetuning zu nennen: 

Alle Hyperparameter der Optimierungsfunktion haben einen technischen Ursprung, und ein Verständnis der Algorithmen ist notwendig, um die Parameter sinnvoll zu wählen. Zwar sind die Standardeinstellungen hier meistens hinreichend, allerdings können durch bewusste Anpassung ebenfalls markante Besserungen in der Trainingsgeschwindigkeit erzielt werden. 

So zeigten höhere Lern- und Verfalls-Parameter bei den weitestgehend homogenen Trinkgelddaten ein wesentlich schnelleres Terminieren das Trainings bei gleichbleibender Genauigkeit. 

Für andere Experimente, z.B. dem Fahrtenaufkommen, zeigten diese Einstellungen katastrophale Ergebnisse - die Daten waren einfach zu verschieden, um mit diesen Parametern behandelt zu werden. 

~\newline Große Teile sind allerdings rein experimentell - vor Allem die optimale Größe der einzelnen Hidden Layer zu finden gestaltet sich im Wesentlichen durch Versuche.

~\newline Insgesamt kann zwar jedes Problem rein experimentell gelöst werden, allerdings stellt dies oft mehr einen \textit{Glückstreffer} dar. 

Ein Verständnis der Algorithmen, ihrer Eigenschaften und der verwendeten Funktionen ermöglicht nicht nur eine bessere Ausgangslage für das Tuning, sondern ist eine Grundvoraussetzung für die Fehleranalyse und Problembehebung. 

~\newline Insofern ist der Ansatz, die Herangehensweise \textit{a priori} als Kunst zu bezeichnen nicht gerechtfertigt. Die Arbeit mit neuronalen Netzen entpuppte sich als gesunde Mischung aus Kunst und Wissenschaft. 

\paragraph{Ausblick}~\newline Innerhalb der Arbeit ließen sich zwei weiterführende Themen ausmachen: 

Der Fokus auf Alternativen bei der Umsatzprognose sowie die Übertragung und Verallgemeinerung der erarbeiteten Netze. Diese beiden Ansätze werden kurz ausgearbeitet. 

~\newline Die Umsatzprognose stellt nach wie vor den wichtigsten Use-Case für viele Unternehmen dar, und konnte ihm Rahmen dieser Arbeit mit neuronalen Netzen nicht gelöst werden. Im Zuge einer weiteren Arbeit kann der Hauptfokus auf der Umsatzprognose liegen, und verschiedene Formen der Vorhersage benutzt und verglichen werden um Ergebnisse zu erzielen. 

Möglichkeiten stellen hierbei klassische statistische Methoden dar, oder eine Vertiefung anderer technischer Ansätze wie etwa Timeseries-Datenbanken oder Modellierung innerhalb einer Mehrdimensionalen-Datenbank.

~\newline Ein weiterer Punkt ist die Portierung der in dieser Arbeit erzeugten neuronalen Netze auf ähnliche Anwendungsgebiete, konkreter auf Taxidaten einer anderen Stadt. 

Hierbei zu lösen sind Schwierigkeiten bezüglich unterschiedlicher Eingabevektoren, vergleichbare Tests zu erzeugen sowie die technische Umsetzung innerhalb des SQL-Servers.  
\newpage\newpage

% ---------------------------- Literaturverzeichnis ----------------------------------------------

\bibliographystyle{plain}
\bibliography{Quellen/src}


% ------------------------------- Anhang ---------------------------------------------------------

\begin{appendix}
\clearpage
\pagenumbering{Roman}						% römische Seitenzahlen für Anhang
\end{appendix}


\end{document}
